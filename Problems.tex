\documentclass{article}
\usepackage{amsmath,amscd,amssymb,latexsym, amsfonts}
%\usepackage{mathtools}
\usepackage{amsthm}

\theoremstyle{plain}
\newtheorem{iassumption}{Assumption}
\newtheorem{theorem}{Theorem}[section]
\newtheorem{conjecture}[theorem]{Conjecture}
\newtheorem{proposition}[theorem]{Proposition}
\newtheorem{corollary}[theorem]{Corollary}
\newtheorem{hypothesis}[theorem]{Hypothesis}
\newtheorem{assumption}[theorem]{Assumption}
\newtheorem{lemma}[theorem]{Lemma}
\newtheorem{question}[theorem]{Question}
\newtheorem{exercise}[theorem]{Exercise}
\newtheorem{statement}[theorem]{Statement}
\newtheorem{example}[theorem]{Example}

\theoremstyle{definition}
\newtheorem{definition}[theorem]{Definition}
\newtheorem{notation}[theorem]{Notation}
\newtheorem{remark}[theorem]{Remark}

\DeclareMathOperator{\Gal}{Gal}
\DeclareMathOperator{\val}{val}
\DeclareMathOperator{\HH}{H}
\DeclareMathOperator{\Ad}{Ad}
\DeclareMathOperator{\Nm}{Nm}
\DeclareMathOperator{\Hom}{Hom}
\DeclareMathOperator{\Spec}{Spec}
\DeclareMathOperator{\Res}{Res}
\DeclareMathOperator{\Fr}{Fr}
\DeclareMathOperator{\Tr}{Tr}
\DeclareMathOperator{\Ind}{Ind}
\DeclareMathOperator{\gen}{gen}

\DeclareMathOperator{\GL}{GL}
\DeclareMathOperator{\PGL}{PGL}
\DeclareMathOperator{\SL}{SL}

\newcommand{\mat}[4]{\left( \begin{array}{cc} {#1} & {#2} \\ {#3} & {#4}
\end{array} \right)}
\newcommand{\TT}{\mathcal{T}}
\newcommand{\C}{\mathcal{C}}
\newcommand{\RR}{\mathbb{R}}
\newcommand{\CC}{\mathbb{C}}
\newcommand{\CCx}{\mathbb{C}^\times}
\newcommand{\OK}{\mathcal{O}_K}
\newcommand{\OKn}{\mathcal{O}_{K_n}}
\newcommand{\pK}{\mathfrak{p}_K}
\newcommand{\pL}{\mathfrak{p}_L}
\newcommand{\OL}{\mathcal{O}_L}
\newcommand{\ZZ}{\mathbb{Z}}
\newcommand{\QQ}{\mathbb{Q}}
\newcommand{\Qp}{\mathbb{Q}_p}
\newcommand{\Gm}{\mathbb{G}_m}
\newcommand{\Lx}{L^\times}

\newcommand{\Weil}{\mathcal{W}}
\newcommand{\WD}{\mathcal{W}'}
\newcommand{\Lpack}{\mathcal{L}}
\newcommand{\Pgen}{P_G^{\gen}}
\newcommand{\bmu}{\boldsymbol\mu}
\newcommand{\mugen}{\bmu^{\gen}}

\newcommand{\la}{\langle}
\newcommand{\ra}{\rangle}

\newcommand{\invlim}[1]{\varprojlim_{#1}}
\newcommand{\Normalizer}[2]{\operatorname{N}_{#2}(#1)}

\begin{document}

\section{Problems}

\begin{enumerate}

\item For $K$ finite over $\Qp$ and $G$ a unitary group over $K$ splitting over $E$, classify maximal tori in $G$ in terms of \'etale algebras over $E$ with involution.

\begin{conjecture}
They should correspond to \'etale algebras $F / E$ with involution $\sigma$ restricting to the nontrivial involution of $E / K$.  Write $T_F$ for the corresponding torus, which is restriction of scalars of the Norm 1 torus from $F$ to $F^\sigma$.
\end{conjecture}
\item Given $T_F$, which unitary groups does it embed into.

\begin{conjecture}
Given $\kappa \in F^\sigma$, we can construct a hermitian space from the form $\la x, y \ra = \Tr_{F/E}(\kappa \cdot x \sigma(y)$ for $x, y \in F$.  These are precisely the unitary groups into which $T_F$ embeds.
\end{conjecture}

\item What are the invariants of these forms that we've constructed?

\begin{conjecture}
Taking the discriminant of this form should induce a homomorphism from $(F^\sigma)^\times$ to $K^\times / \Nm_{E/K} E^\times$ with kernel $\Nm_{F/F^\sigma} F^\times$.  Have we actually normalized correctly to make this happen.
\end{conjecture}

\item From this description can we read off various properties of the torus: ramified vs unramified, anisotropic, character and cocharacter lattices, N\'eron model....

\item Can we give both the cohomological and intrinsic formulation of these questions?

\item What is the global picture?  Are the local-global obstructions?

\item What about $\RR$?

\item For anisotropic tori can you describe the reduction of the maximal compact containing this torus?

\end{enumerate}



\end{document}