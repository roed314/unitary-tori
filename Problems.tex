\documentclass{article}
\usepackage{amsmath,amscd,amssymb,latexsym, amsfonts}
\usepackage{mathtools}
\usepackage{amsthm}

\theoremstyle{plain}
\newtheorem{iassumption}{Assumption}
\newtheorem{theorem}{Theorem}[section]
\newtheorem{conjecture}[theorem]{Conjecture}
\newtheorem{proposition}[theorem]{Proposition}
\newtheorem{corollary}[theorem]{Corollary}
\newtheorem{hypothesis}[theorem]{Hypothesis}
\newtheorem{assumption}[theorem]{Assumption}
\newtheorem{lemma}[theorem]{Lemma}
\newtheorem{question}[theorem]{Question}
\newtheorem{exercise}[theorem]{Exercise}
\newtheorem{statement}[theorem]{Statement}
\newtheorem{example}[theorem]{Example}

\theoremstyle{definition}
\newtheorem{definition}[theorem]{Definition}
\newtheorem{notation}[theorem]{Notation}
\newtheorem{remark}[theorem]{Remark}

\DeclareMathOperator{\Gal}{Gal}
\DeclareMathOperator{\val}{val}
\DeclareMathOperator{\HH}{H}
\DeclareMathOperator{\Ad}{Ad}
\DeclareMathOperator{\Nm}{Nm}
\DeclareMathOperator{\Hom}{Hom}
\DeclareMathOperator{\Spec}{Spec}
\DeclareMathOperator{\Res}{Res}
\DeclareMathOperator{\Fr}{Fr}
\DeclareMathOperator{\Tr}{Tr}
\DeclareMathOperator{\Ind}{Ind}
\DeclareMathOperator{\gen}{gen}

\DeclareMathOperator{\GL}{GL}
\DeclareMathOperator{\PGL}{PGL}
\DeclareMathOperator{\SL}{SL}

\newcommand{\mat}[4]{\left( \begin{array}{cc} {#1} & {#2} \\ {#3} & {#4}
\end{array} \right)}
\newcommand{\TT}{\mathcal{T}}
\newcommand{\C}{\mathcal{C}}
\newcommand{\RR}{\mathbb{R}}
\newcommand{\CC}{\mathbb{C}}
\newcommand{\CCx}{\mathbb{C}^\times}
\newcommand{\OK}{\mathcal{O}_K}
\newcommand{\OKn}{\mathcal{O}_{K_n}}
\newcommand{\pK}{\mathfrak{p}_K}
\newcommand{\pL}{\mathfrak{p}_L}
\newcommand{\OL}{\mathcal{O}_L}
\newcommand{\ZZ}{\mathbb{Z}}
\newcommand{\QQ}{\mathbb{Q}}
\newcommand{\Qp}{\mathbb{Q}_p}
\newcommand{\Gm}{\mathbb{G}_m}
\newcommand{\Lx}{L^\times}

\newcommand{\Weil}{\mathcal{W}}
\newcommand{\WD}{\mathcal{W}'}
\newcommand{\Lpack}{\mathcal{L}}
\newcommand{\Pgen}{P_G^{\gen}}
\newcommand{\bmu}{\boldsymbol\mu}
\newcommand{\mugen}{\bmu^{\gen}}

\newcommand{\la}{\langle}
\newcommand{\ra}{\rangle}

\newcommand{\invlim}[1]{\varprojlim_{#1}}
\newcommand{\Normalizer}[2]{\operatorname{N}_{#2}(#1)}

\begin{document}

\section{Problems}

\begin{enumerate}

\item For $K$ finite over $\Qp$ and $G$ a unitary group over $K$ splitting over $E$, classify maximal tori in $G$ in terms of \'etale algebras over $E$ with involution.

\begin{conjecture}
They should correspond to \'etale algebras $F / E$ with involution $\sigma$ restricting to the nontrivial involution of $E / K$.  Write $T_F$ for the corresponding torus, which is restriction of scalars of the Norm 1 torus from $F$ to $F^\sigma$.
\end{conjecture}
\item Given $T_F$, which unitary groups does it embed into.

\begin{conjecture}
Given $\kappa \in F^\sigma$, we can construct a hermitian space from the form $\la x, y \ra = \Tr_{F/E}(\kappa \cdot x \sigma(y)$ for $x, y \in F$.  These are precisely the unitary groups into which $T_F$ embeds.
\end{conjecture}

\item What are the invariants of these forms that we've constructed?

\begin{conjecture}
Taking the discriminant of this form should induce a homomorphism from $(F^\sigma)^\times$ to $K^\times / \Nm_{E/K} E^\times$ with kernel $\Nm_{F/F^\sigma} F^\times$.  Have we actually normalized correctly to make this happen.
\end{conjecture}

\item From this description can we read off various properties of the torus: ramified vs unramified, anisotropic, character and cocharacter lattices, N\'eron model....

\item Can we give both the cohomological and intrinsic formulation of these questions?

\item What is the global picture?  Are the local-global obstructions?

\item What about $\RR$?

\item For anisotropic tori can you describe the reduction of the maximal compact containing this torus?

\end{enumerate}

\section{Tori and \'Etale Algebras}

Let $T$ be a torus defined over $K$ and $\rho:T_K \rightarrow \GL(V_K)$ a $K$-rational algebraic representation of $T$ (one can and might as well assume $\rho$ is injective).

Associated to this data there exists an \'etale algebra $E$ over $K$, inclusions $\rho'' : E \hookrightarrow End(V_K)$ and $\rho' : T \rightarrow T_E$ for which $\rho = \rho''\circ\rho'$.

\begin{proposition}
We have the following constructions of $E$:
\begin{enumerate}
\item Let $X$ denote the set of characters of $T$ appearing via the representation of $\rho$ over $\overline{K}$. This is a Galois set for the action of $\Gal(\overline{K}/K)$ hence defines a unique \'etale algebra $E$.
\item Let $C = C_\rho(T)$ be the centralizer of the image of $\rho(T)$ in $End(V_K)$ (as a scheme over $K$).
           Let $Z = Z(C_\rho(T))$ be the centre of $C$ (as a scheme over $K$).
           Let $E = Z(K)$.
\item If the characters $\chi$ appearing in $X$ are distinguished by their values on $K$ points then $E$ is the $K$-span in $End(V_K)$ of $\rho(T(K))$.
\end{enumerate}
\end{proposition}
\begin{proof}
That (2) and (3) give \'etale algebras follows from base change to the algebraic closure.
That the algebra constructed by (3) is contained in that constructed by (2) is clear.
The converse can be checked over the algebraic closure, using flatness and dimension considerations.

The equivalence of (1) and (3) follows from the characterization of \'etale algebras by Galois descent.
\end{proof}

\begin{remark}
If the size of $X$ equivalently the $K$-dimension of $E$ equals the $K$ dimension of $V_K$ then $V_K$ is a rank $1$ module over $E$ hence may be (non-canonically) identified with $E$.
\end{remark}

\subsection{Structures on $E$}

The choice of representation $\rho$ determines structures on $E$.

\begin{proposition}
Let $B : V_K \times V_K \rightarrow K$ be a non-degenerate bilinear form preserved by the image of $\rho$ then the adjoint involution on $End(V_K)$ (relative to $B$) defines an involution $\sigma$ on $E$.
Moreover, $T \rightarrow T_{E\sigma} \rightarrow T_E$ is a factorization of the map $\rho'$.
\end{proposition}
\begin{proof}
Denote by $\sigma$ the adjoint involution on $End(V_K)$.
We are given that $B(x,y) = B(\rho(t)x,\rho(t)y) = B(x,\sigma(\rho(t))\rho(t)y)$ for all $x$ and $y$. By the non-degeneracy of $B$ this implies $\sigma(\rho(t))\rho(t)$ is the identity and thus $\sigma(\rho(t)) = \rho(t)^{-1}$. Thus, $\sigma$ takes $T$ to $T$. by the naturality of the construction of $E$ above $\sigma$ takes $E$ to $E$.

It is now clear that $\rho(T) \subset T_{E,\sigma}$.
\end{proof}



\begin{proposition}
Let $A$ be an algebra over $K$ and suppose $V_K$ has the structure of an $A$-module.
Suppose $\rho(T)$ embeds into $End_A(V_K)$ then $E\hookrightarrow End_A(V_K)$.

Suppose $\rho(T)$ commutes with the action of $A$, then the image of $E$ commutes with the action of $A$.
If moreover the $K$-dimension of $E$ equals that of $V$ this also gives $E$ the structure of an $A$-algebra.
\end{proposition}
\begin{proof}
By naturality of construction if $\rho(T)$ acts by $A$ endomorphisms then so too does $E$.
Likewise if $\rho(T)$ commutes with the action of $A$, then so too does $E$.

Finally, if $K$-dimensions are equal, then by identifying $V_K$ with $E$ we have given $E$ the structure of an $A$-module. Since the $A$ and $E$ multiplication commute, we are giving $E$ the structure of an $A$-algebra.
\end{proof}
\begin{remark}
The case of non-commutative $E$ is mostly relevant to inner forms of symplectic groups.
In what follows we will assume $A$ is a commutative algebra over $K$.
\end{remark}

\begin{proposition}
Let $A$ be a quadratic algebra over $K$. Suppose $V_K$ has the structure of an $A$-module, and $B : V_K \times V_K \rightarrow A$ is a non-degenerate Hermitian form.
Suppose  $\rho(T)$ preserves $B$ then $E\hookrightarrow End_A(V_K)$, the image of $E$ commutes with the action of $A$,  $E$ has the structure of an $A$-algebra, and the involution on $E$ restricts to that on $A$.
\end{proposition}
\begin{proof}
The only new statement here is that the involution on $E$ restricts to the involution on $A$.
The involution on $E$ was induced by the adjoint from $B$. But the involution on $A$ is by construction the adjoint.
\end{proof}
\begin{remark}
The $K$-algebra $E= A\times A$ has two meaningfully different $A$-algebra structures arising from the inclusions:
\[ a \mapsto (a,a) \qquad \text{and}\qquad a\mapsto (a,\sigma(a)) \]
once we impose an involution on $E$. The two cases are meaningfully distinguished by how the involution restricts to $A$.
\end{remark}


\subsection{Sources of Representations}

Let $G$ be an algebraic group over $K$ and suppose $T \hookrightarrow G$.
Then any representation $\rho$ of $G$ restricts to a representation $\rho$ of $T$.

The structures imposed on $E_T$ by $\rho$ are determined by the representation of $G$ and not by $T$.

\begin{example}
Let $\rho:G \hookrightarrow \GL(V_K,B)$ be a representation preserving the bilinear form $B$ then for all tori $T$ in $G$, there exists an \'etale algebra with involution $(E,\sigma)$ and and inclusion $T \hookrightarrow T_{E,\sigma}$.

This holds even if $T$ is a split torus in the split unitary group $G=GL_n$ and $\rho$ is the representation corresponding to the standard representation of the unitary group.
\end{example}

\begin{example}
If the form $B$ above is Hermitian for an action of $A$ on $V_K$ then $E$ is an $A$-algebra with the involution on $E$ restricting to that on $A$.

In the split case the algebra $A$ is $K\oplus K$.
\end{example}

\section{Forms preserved by $T_{E,\sigma}$}

\subsection{Structures of the Forms}

\begin{theorem}
Suppose $B_1,B_2$ be bilinear (or Hermitian) forms on $E$ for which the adjoint maps $Ad_{B_1}$ and $Ad_{B_2}$ induce the same map $\sigma:E\rightarrow E$.
That is $B_i(ex,y) = B_i(x,\sigma(e)y)$, ie, we are in the case where $T_{E,\sigma}$ preserves $B_1$ and $B_2$.
Then, $B_1(x,y) = B_2(\lambda x,y)$.
\end{theorem}
\begin{proof}
Both $B_1$ and $B_2$ induce isomorphism between $E$ and its K (resp. A)-linear dual.
Hence, there exists a unique $\lambda\in E$ such that $B_1(1,y) = B_2(\lambda,y)$ for all $y\in E$.
We have:
\[ B_1(e,x) = B_1(1,\sigma(e)x) = B_2(\lambda,\sigma(e)x) = B_2(\lambda e,x). \]
\end{proof}
\begin{theorem}
If in the above if $B_1$ and $B_2$ are symmetric then $\sigma(\lambda)=\lambda$.

If in the above $B_1$ and $B_2$ are Hermitian then $\sigma(\lambda)=\lambda$.
\end{theorem}
\begin{proof}
We have the following calculation:
\[ B_2(\lambda y,x) = B_1(y,x) = B_1(x,y) = B_2(\lambda x,y) = B_2(x,\sigma(\lambda) y) = B_2(\sigma(\lambda)y,x) \]
Alternatively:
\[ B_2(\lambda y,x) = B_1(y,x) = \overline{B_1(x,y)} = B_2(\lambda x,y) = \overline{B_2(x,\sigma(\lambda) y)} = B_2(\sigma(\lambda)y,x) \]
And this holds for all $x,y$.
Hence, by the uniqueness of $\lambda$ from above, $\lambda=\sigma(\lambda)$.
\end{proof}
\begin{corollary}
All the non-degenerate symmetric bilinear forms on $E$ are of the form:
\[ Tr_{E/K}(\lambda x\sigma(y) \]
All the non-degenerate Hermitan forms on $E$ are of the form:
\[ Tr_{E/A}(\lambda x\sigma(y) \]
where $\lambda \in E^\sigma$ is a unit.
In both cases, the isomorphism class of the form depends only on $\lambda \in (E^\sigma)^\times/N_{E/E^\sigma}(E^\times)$.
\end{corollary}

\subsection{Invariants of Hermitian Forms}

TODO - in this section we might benefit from renormalizing the form, in my experiance letting $\lambda' = \lambda/f'_z(z)$ where $z\in E^\sigma$ is such that $\sqrt{z}$ primitively generates $E$ over $K$ is a good choice.
(Such $z$ exist outside characteristic $2$ when $E$ is a field, or $E$ is an \'etale algebra and $K$ infinite).

\begin{notation}
Given an \'etale algebra with involution $(E,\sigma)$ and $\lambda \in (E^\sigma)^\times$
denote the Hermitian form:
\[ H_{E,\sigma,\lambda} := Tr_{E/A}(\lambda x\sigma(y)) \]
and the symmetric bilinear form:
\[ B_{E,\sigma,\lambda} := Tr_{E/A}(\lambda x\sigma(y)). \]
\end{notation}

\begin{proposition}
Let $H$ be any Hermitian form and denote:
\[ B(x,y) = Tr_{A/K}(H(x,y)). \]
Then:
\begin{enumerate}
\item The discriminant of $H$ (and the discriminant of $A$) determines the Hasse invariant of $B$. (The converse holds whenever $A$ is not split.)
\item The discriminant of $A$ determines the discriminant of $B$.
\item The signatures of $H$ (and the infinite ramification of $A$) determines the signatures of $B$. The converse holds if $A$ is ramified.
\end{enumerate}
In particular the invariants of $B$ are equivalent to those of $H$.
\end{proposition}
\begin{proof}
TODO - complete sketch.

Hasse invariant uses formulas from my paper (see also BCKM).

discriminant is easy (but also in my thesis)

signature is easy by checking cases.
\end{proof}

\begin{proposition}
The Hasse invariant of $B_{E,\sigma,\lambda}$ is:
\[ TODO \]
\end{proposition}

\begin{proposition}
The discriminant of invariant of $H_{E,\sigma,\lambda}$ is:
\[ TODO \]
Moreover, the map:
\[ (E^\sigma)^\times/N_{E/E^\sigma}(E^\times) \rightarrow K^\times/N_{A/K}(A^\times) \]
is a homomorphism.
\end{proposition}
\begin{proof}
TODO - complete sketch

We know Hasse invariant of $B_{E,\sigma,\lambda}$, the same computation (plus small amount of extra work) lets us compute Hasse invariant of $Tr_{A/K}(dx_0^2 \oplus_i x_i^2)$.
Both formulas should be in terms of the Hilbert symbol.

Provided $E/E^\sigma$ (equiv $A/K$) not split the formula should tell us non-trivial $d$ come from non-trivial $\lambda$.

Note that  $(E^\sigma)^\times/N_{E/E^\sigma}(E^\times)$ is $( K^\times/N_{A/K}(A^\times))^\ell$ where $\ell$ is the number of field factors of $E$.
\end{proof}

\begin{proposition}
The signature of $H_{E,\sigma,\lambda}$ is:
\[ TODO \]
\end{proposition}

\subsection{Existence of Forms}

\begin{theorem}
Given $H$ a Hermitian form and $(E,\sigma)$ an \'etale algebra with involution (of the second kind) over $A$.Then for all $p$ primes of $K$ there exists $\lambda_p \in E^\sigma$ with:
\[ H_p \simeq H_{E_p,\sigma,\lambda_p} \]
TODO - I am pretty sure there are no conditions, I haven't fully checked.
\end{theorem}
\begin{proof}
TODO - complete sketch

non-split Infinite places (use signature proposition)

non-split finite places (use discriminant map proposition)

Split places (trivial)
\end{proof}


\begin{theorem}
TODO - phrasing

There exists $\lambda_p$ locally everywhere if and only if there exists $\lambda_p$ globally.
\end{theorem}
\begin{proof}
TODO - complete sketch (also check that this is true).

Argument 1:

The local global obstruction allowing $\lambda$ such that $B_{E,\sigma,\lambda} \simeq B$ does not exist because for any place $p$ where $A/K$ is innert, then for every field factor of $E$ we have $E/E^\sigma$ is innert, thus there is no local global obstruction.

The equivalence of $B$'s and $H$'s should conclude the result.

\bigskip
Argument 2:

Optimistically global class field theory has a name for what we are doing.
We need a $\lambda$ with controlled number of positive/negative embeddings, asside from this we are trying to say that the map:
\[ (E^\sigma)^\times/N_{E/E^\sigma}(E^\times) \rightarrow K^\times/N_{A/K}(A^\times) \]
is surjective.
So in total what we want is surjective when we appropriately restrict to `totally postive' elements. This sounds like a very standard CFT type problem, and since by Argument 1 this result should hold... this result should have been proven by someone else before now.
\end{proof}


\section{Properties of $T_{E,\sigma}$}

\subsection{Anisotropic}

\begin{proposition}
The torus $T_{E,\sigma}$ is anisotropic if all the field factors of $E$ are $\sigma$-stable.
\end{proposition}
\begin{proof}
TODO-
\end{proof}

\begin{example}
Consider $E = A \times A$ where $\sigma$ acts on each factor.
This has the structure of an $A$ algebra, where $\sigma$ restricts to $\sigma$, by the inclusion $A\hookrightarrow A\times A$:
\[ a \mapsto (a,a) \]
The torus $T_{E,\sigma}$ is anisotropic.

\bigskip
Consider $E = A \times A$ where $\sigma$ acts to interchange factors.
Give this the structure of an $A$ algebra, where $\sigma$ restricts to $\sigma$, by the inclusion $A\hookrightarrow A\times A$:
\[ a \mapsto (a,\sigma(a)) \]
The torus $T_{E,\sigma}$ is not anisotropic.
\end{example}


\subsection{Unramified}

\begin{proposition}
The torus $T_{E,\sigma}$ is unramified if all the field factors of $E$ are unramified
\end{proposition}
\begin{proof}
TODO-
\end{proof}


\subsection{``Relatively'' Unramified}

Your special unramifiedness when $A/K$ was ramified.

\section{Cohomology}

This is stuff I don't recall super well, I think the following is roughly true, you might recall better than I.

\begin{question}
The set of pairs $(T,U)$ consisting of $T$ a maximal torus (up to rational conjugacy) of $U$ and $U$ a pure inner form of a unitary group are in bijection with:
\[ H^1( \Gal(\overline{K}/K), N_{U_0}(T_0) ) \]
\end{question}
If $T_0$ is split this becomes:
\[ H^1( \Gal(\overline{K}/K), W(U,T_0) ) \]

\begin{question}
The set of pure inner forms of $U$ containing a torus isomorphic to $T_0$ is in bijection with the kernel of the natural map:
\[ H^1( \Gal(\overline{K}/K), N_{U_0}(T_0) ) \rightarrow H^1( \Gal(\overline{K}/K),N_{GL_n}(T_E )) \]
\end{question}

\begin{question}
the set of maximal tori $T$ in $U_0$ (up to rational conjugacy) is in bijection with the kernel of the natural map:
\[  H^1( \Gal(\overline{K}/K), N_{U_0}(T_0) ) \rightarrow H^1( \Gal(\overline{K}/K), U_0) \]
\end{question}

The above should somehow give us that:
\begin{question}
The forms of $T$ contained in $U$ are in bijection with:
\[ H^1(\Gal(\overline{K}/K), W(U_0,T_0)^{\Gal(\overline{K}/K} ) \]
\end{question}
However this could plausibly require that $T_0$ be split or that $U_0$ is quasi-split.










\end{document}