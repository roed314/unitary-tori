\documentclass{article}
\usepackage{amsmath,amscd,amssymb,latexsym, amsfonts}
%\usepackage{mathtools}
\usepackage{amsthm}
\usepackage{xypic}
\usepackage{yfonts, mathrsfs}

\theoremstyle{plain}
\newtheorem{iassumption}{Assumption}
\newtheorem{theorem}{Theorem}[section]
\newtheorem{conjecture}[theorem]{Conjecture}
\newtheorem{proposition}[theorem]{Proposition}
\newtheorem{corollary}[theorem]{Corollary}
\newtheorem{hypothesis}[theorem]{Hypothesis}
\newtheorem{assumption}[theorem]{Assumption}
\newtheorem{lemma}[theorem]{Lemma}
\newtheorem{question}[theorem]{Question}
\newtheorem{exercise}[theorem]{Exercise}
\newtheorem{statement}[theorem]{Statement}
\newtheorem{example}[theorem]{Example}

\theoremstyle{definition}
\newtheorem{definition}[theorem]{Definition}
\newtheorem{notation}[theorem]{Notation}
\newtheorem{remark}[theorem]{Remark}

\DeclareMathOperator{\Aut}{Aut}
\DeclareMathOperator{\Gal}{Gal}
\DeclareMathOperator{\val}{val}
\DeclareMathOperator{\HH}{H}
\DeclareMathOperator{\Ad}{Ad}
\DeclareMathOperator{\Nm}{Nm}
\DeclareMathOperator{\Hom}{Hom}
\DeclareMathOperator{\Spec}{Spec}
\DeclareMathOperator{\Res}{Res}
\DeclareMathOperator{\Fr}{Fr}
\DeclareMathOperator{\Tr}{Tr}
\DeclareMathOperator{\Ind}{Ind}
\DeclareMathOperator{\gen}{gen}
\DeclareMathOperator{\End}{End}
\DeclareMathOperator{\Cor}{Cor}

\DeclareMathOperator{\GL}{GL}
\DeclareMathOperator{\PGL}{PGL}
\DeclareMathOperator{\SL}{SL}

\newcommand{\mat}[4]{\left( \begin{array}{cc} {#1} & {#2} \\ {#3} & {#4}
\end{array} \right)}
\newcommand{\TT}{\mathcal{T}}
\newcommand{\C}{\mathcal{C}}
\newcommand{\RR}{\mathbb{R}}
\newcommand{\CC}{\mathbb{C}}
\newcommand{\CCx}{\mathbb{C}^\times}
\newcommand{\OK}{\mathcal{O}_K}
\newcommand{\OKn}{\mathcal{O}_{K_n}}
\newcommand{\pK}{\mathfrak{p}_K}
\newcommand{\pL}{\mathfrak{p}_L}
\newcommand{\OL}{\mathcal{O}_L}
\newcommand{\ZZ}{\mathbb{Z}}
\newcommand{\QQ}{\mathbb{Q}}
\newcommand{\Qp}{\mathbb{Q}_p}
\newcommand{\Gm}{\mathbb{G}_m}
\newcommand{\Lx}{L^\times}
\newcommand{\fp}{\mathfrak{p}}

\newcommand{\GalKbK}{\Gamma_K}
\newcommand{\Kb}{\overline{K}}

\newcommand{\Weil}{\mathcal{W}}
\newcommand{\WD}{\mathcal{W}'}
\newcommand{\Lpack}{\mathcal{L}}
\newcommand{\Pgen}{P_G^{\gen}}
\newcommand{\bmu}{\boldsymbol\mu}
\newcommand{\mugen}{\bmu^{\gen}}

\newcommand{\la}{\langle}
\newcommand{\ra}{\rangle}

%% Labeled items
\makeatletter
\newcommand{\labitem}[2]{
\def\@itemlabel{\textbf{#1}}
\item
\def\@currentlabel{#1}\label{#2}}
\makeatother

\newcommand{\invlim}[1]{\varprojlim_{#1}}
\newcommand{\Normalizer}[2]{\operatorname{N}_{#2}(#1)}

\newcommand{\TODO}[1]{\textbf{TODO-#1}}

\newcommand{\aK}{K}
\newcommand{\gK}{\mathcal{K}}
\newcommand{\lK}{\mathfrak{K}}
\newcommand{\fK}{\textswab{K}}
\newcommand{\aL}{L}
\newcommand{\gL}{\mathcal{L}}
\newcommand{\lL}{\mathfrak{L}}
\newcommand{\fL}{\textswab{L}}
\newcommand{\aE}{E}
\newcommand{\gE}{\mathcal{E}}
\newcommand{\lE}{\mathfrak{E}}
\newcommand{\fE}{\textswab{E}}
\newcommand{\aV}{V}
\newcommand{\gV}{\mathcal{V}}
\newcommand{\lV}{\mathfrak{V}}
\newcommand{\fV}{\textswab{V}}

\title{Algebraic Tori in Unitary Groups}
\author{Andrew Fiori \& David Roe}

\begin{document}

\maketitle

\begin{abstract}
We give a characterization of the algebraic tori in unitary groups in terms of the structure of \'etale algebras with involution. We determine relavent properties, such as unramifiedness or anisotropicness, of the tori from the associated algebra. We discuss the implications on the structure of the building for the unitary group. We give a cohomological reinterpretation of the results.
\end{abstract}

\tableofcontents

\section{Introduction}

The problem of giving a classification and/or characterization of algebraic tori in algebraic groups has several sources of motivation coming out of algebraic number theory. In particular it has applications in the the theory of automorphic forms, Shimura varieties and the Langlands conjectures. Moreover, the algebraic tori play a key role in the construction of the building associated to the group, and also in the classification of forms of algebraic groups. Beyond all this the question is natural from a theoretical point of view in the theory of algebraic groups.  

\TODO{More Motivational nonsense}

\TODO{References to important literature}

\TODO{Summary of Layout}

\section{Definitions, Notation and Background}

\TODO{} Decide which of the following to define/give notations for, we probably don't need all of it. Some of it can be postponed until the sections where it is used.
\begin{itemize}
\item Explain our notation for fields and \'etale algebras: $\aK$, $\gK$, $\lK$, $\fK$, $\aL$, $\gL$, $\lL$, $\fL$, $\aE$, $\gE$, $\lE$, $\fE$, $\aV$, $\gV$, $\lV$, $\fV$.
\item The setting is a base field $K$ taken to be a $p$-adic field or number field.
\item We shall denote by $\GalKbK$ the absolute Galois group of $K$.


\end{itemize}

\subsection{Tori and \'Etale Algebras}

By an algebraic torus over $K$ we mean a group scheme $T$ over $K$ which is isomorphic to ${\Gm}_{\overline{K}}^n$ over $\overline{K}$.
The automorphism group of $T$ over $\overline{K}$ is $GL_n(\ZZ)$ and thus forms of $T$ are classified by the cohomology set $H^1(\GalKbK,GL_n(\ZZ))$.

By an \'etala algebra $E$ over the field $K$ we mean that $E$ is isomorphic to the direct product of (seperable) field extensions $E_i$ of $K$.
Over the algebraic closure $\overline{K}$ of $K$ we have that $E\otimes_K \overline{K} \simeq \prod_\rho \overline{K}_\rho$ where $\rho \in \Hom(E,\overline{K})$.
It is well known that forms of $E$ are in bijection with forms of the Galois set $\Hom(E,\overline{K})$ from which we obtain that \'etale algebras are classified by the Galois set $H^1(\GalKbK, \Sigma_n)$.

Associated to an \'etale algebra $E$ is an algebraic torus which we shall denote by $T_E = \Res_{E/K} {\Gm}_E$.
Its functor of points is given by:
\[ T_E(R) = \{ g \in (E\otimes_K R)^\times \}. \]
The splitting field of $T_E$ is the normal closure of $E$ and a basis for its character module is naturally identified with $\Hom(E,\overline{K})$.

\begin{remark}
All tori can be realized as subtori of $T_E$ (see \ref{TODO - forward ref}).
\end{remark}

By an \'etale algebra with involution $(E,\sigma)$, we mean an \'etale algebra $E$, and an automorphism of $E$ of order $2$ such that $E^\sigma$ the elements of $E$ fixed by $\sigma$ has half the dimension of $E$.
This is equivalent to specifying the \'etale algebra $E^\sigma$, and a quadratic \'etale extension $E = E^\sigma[x]/(x^2-d)$ of $E^\sigma$ for an element $d\in (E^\sigma)^\times$.

Associated to an \'etale algebra with involution $(E,\sigma)$ is an algebraic torus which we shall denote by $T_{E,\sigma} \subset T_E$.
Its functor of points is:
\[ T_{E,\sigma}(R) = \{ g \in (E\otimes_K R)^\times \mid g\sigma(g) = 1 \}. \]
This torus sits in the exact sequence:
\[ 1 \rightarrow  T_{E,\sigma} \rightarrow T_E \overset{N_{E/E^\sigma}}\rightarrow T_{E^\sigma} \rightarrow 1 \]


\subsection{Hermitian Spaces}

Fix a quadratic \'etale extension $E$ of $K$ and a free $E$-module $V$.
A Hermitian structure on $V$ is a non-degenerate skew form $H : V\times V \rightarrow E$.
We shall consider the group $U(H) = \Aut(V,H)$ and $SU(H) = \Aut(V,H,\det_E)$ where $\det_E$ refers to the $E$-module determinant on $V$.

The forms of $V$ are in bijection with $H^1(\GalKbK, \Aut(V,H))$.
We have an exact sequence:
\[ 1 \rightarrow SU(H) \rightarrow U(H) \rightarrow T_{E,\sigma} \rightarrow 1. \]
This induces an exact sequence:
\[ \ldots \rightarrow H^1(\GalKbK,SU(H))  \rightarrow H^1(\GalKbK,U(H))  \rightarrow H^1(\GalKbK,T_{E,\sigma}) \rightarrow \ldots. \]
We have an isomorphism $H^1(\GalKbK,T_{E,\sigma}) = K^\times/N_{E/K}(E^\times)$ and we shall refer to the associated invariant of $H$ as the discriminant.
The group $SU(H)$ is simply connected, thus its cohomology is supported at the real places of $K$.
When $E$ is split there are no forms, when $E=\C$ forms are classified by their signatures, that is the number of positive and negative eigenvalues for their Grahm matrix.

We summarize the above with the following well known proposition.
\begin{proposition}
Fix a quadratic \'etale extension $E$ of $K$ and a free $E$-module $V$.
Hermitian structures on $E$ are classified up to isomorphism by their discriminants $\delta\in K^\times/N_{E/K}(E^\times)$ and their signatures $(s_\nu,r_\nu)$ at real places of $K$ which ramify in $E$.
Moreover, $\delta$ may be computed as the determinant of the Grahm matrix whereas $s_\nu$ and $r_\nu$ are respectively the number of positive and negative eigenvalues of hte Grahm matrix.
\end{proposition}

\subsection{Unitary Groups}

For our purposes a unitary group is a simply connect group $G$ of type $A_n$.
We may construct such a group as follows.
Fix the following:
\begin{itemize}
\item An quadratic \'etale algebra $E$ over $K$.
\item A central simple $E$-algebra $A$ of degree $n$ with anti-involution $\sigma$ restricting to the non-trivial involution of $E$.
\item An element $J\in F^\sigma$ of norm $\delta$ whose minimal polynomial at a real place $\nu$ of $K$ which ramifies in $E$ has $r_\nu$ positive roots and $s_\nu$ negative roots.
\end{itemize}
Define then $G$ to be the group scheme whose points over a $K$-algebra $R$ are given by:
\[ G(R) = \{ g\in (A\otimes_K R)^\times \mid \sigma(g)Jg = J \text{ and } Nm(g) = 1 \}. \]

We claim that all $G$ arise this way, and that if $n>1$ then $G$ is uniquely determined by the invariants $(E,A,\delta, \left\vert r_\nu-s_\nu\right\vert)$ where $\delta \in K^\times/(K^\times)^nN_{E/K}(E^\times)$ subject only to the condition that $(-1)^{r_\nu-s_\nu}\delta$ is positive at all the real places of $K$ which ramify in $E$.

The forms of $G$ are in bijection with the elements of $H^1(\GalKbK, \Aut(G))$.

When $n>1$ the outer automorphism is a group of order $2$ and we have an exact sequence:
\[ 1 \rightarrow   G^{adj} \rightarrow \Aut(G) \rightarrow \mu_2  \rightarrow 1 \]
from which we obtain an exact sequence:
\[ \ldots  H^1(\GalKbK, G^{adj})  \rightarrow  H^1(\GalKbK, \Aut(G)) \rightarrow  H^1(\GalKbK, \mu_2) \rightarrow \ldots \]
We conclude we can characterize forms of $G$ by invariants in $H^1(\GalKbK, \mu_2)$ and $H^1(\GalKbK, G^{adj})$.

An element of $H^1(\GalKbK, \mu_2)$ corresponds to the quadratic \'etale extension $E$ of $K$.

The center $\mu_{n+1}^\xi)$ of $G$ is the $n+1$ torsion of $T_{E,\sigma}$.
We have an exact sequence:
\[ \ldots \rightarrow H^1(\GalKbK,G) \rightarrow H^1(\GalKbK, G^{adj}) \rightarrow H^2(\GalKbK, \mu_{n+1}^\xi) \rightarrow \ldots. \]
A form of $G$ thus has an invariant in $H^2(\GalKbK, mu_{n+1}^\xi)$.

$H^2(\GalKbK, \mu_{n+1}^\xi)$ fits into the exact sequence:
\[ H^1(\GalKbK, T_{E,\sigma}) \rightarrow H^2(\GalKbK, mu_{n+1}^\xi) \rightarrow H^2(\GalKbK, T_{E,\sigma}) \overset{[n+1]}\rightarrow H^2(\GalKbK, T_{E,\sigma}). \]
We are thus interested in the cohomology of $T_{E,\sigma}$.
The exact sequence $1 \rightarrow T_{E,\sigma} \rightarrow T_E \rightarrow T_{E^\sigma} \rightarrow 1$ and Hilbert's Theorem 90 allow us to compute that 
$H^1(\GalKbK, T_{E,\sigma}) = k^\times/N_{E/k}(E^\times)$ and that $H^2(\GalKbK, T_{E,\sigma})$ is the kernel of the corestriction map from the Brauer group of $E$ to that of $E^\sigma$.
In particular we can now deduce that $H^2(\GalKbK, \mu_{n+1}^\xi)$ corresponds to element a pair of an element of $\delta\in k^\times/(k^\times)^nN_{E/k}(E^\times)$ and a degree $n$ central simple algebra $A$ over $E$ in the $n$-torsion of the Brauer group which is only non-trivial at the places of $E$ over $K$ which split, and for places which split the algebra over one factor is isomorphic to the opposite algebra of that over the other.

It is precisely these algebras $A$ which admit non-trivial involutions acting non-trivially on there center.
Indeed, in this setting we have that $A \otimes_{E,\sigma} E \simeq A^{op}$, that is if we twist the underlying $E$ algebra structure of $A$ then the resulting algebra is isomorphic to the opposite algebra of $A$ as an $E$-algebra. Let $\tau$ be the composition of the maps $A \rightarrow A \otimes_{E,\sigma} E \simeq A^{op} \rightarrow A$. This is an anti-involution on $A$, and as the map $A \rightarrow A \otimes_{E,\sigma} E $ is not a map of $E$-algebras (only of $K$-algebras) the map cannot act trivially on $E$. It must however preserve $E$ as this is the center of $A$. Thus $\tau$ restricts to $\sigma$ on $E$.

\begin{remark}
The choice of $J\in F^\sigma$ is equivalent to choosing the involution $\tau$.
In particular, if we define $\tau'(x) = J\tau(x)J^{-1}$ then we obtain the same group using $\tau'$ and taking for $J$ the identity.

Indeed, by \cite[Prop. 2.18]{The Book of Involutions TODO} all involutions on $A$ arise this way.
Thus, the choice of $J$, can thus be thought of as choosing the involution.
\end{remark}

\begin{remark}
It should be pointed out that the group $K^\times/(K^\times)^nN_{E/K}(E^\times)$ is trivial if either $n$ is odd or if $E = K\times K$ is split.
Further, for local fields at least one of $K^\times/(K^\times)^nN_{E/K}(E^\times)$ or $H^2(\GalKbK, T_{E,\sigma}) $ is trivial.
\end{remark} 

Finally, we must consider the group $H^1(\GalKbK,G)$. As $G$ is simply connect, this is only non-trivial at the real places of $K$.
In terms of real places, there are two cases to consider.
If $E=\C$, then $G$ is the automorphism group of a Hermitian space and $H^1(\GalKbK,G)$ classifies the Hermitian forms of the same discriminant as that defining $G$, as above, over $\RR$ this means it classifies the signature.
When $E=\RR$ then $G$ is the automorphism group of a free module over a division algebra, there are no forms of these hence $H^1(\GalKbK,G)$ is trivial. 

TODO book of involutions 29.20 says this is trivial.
TODO --- Find reference for this ----

\begin{remark}
Because we are dealing with non-abelian Galois cohomology when analyzing exact sequences we must twist by the cokernel elements before considering the structure of the kernel. It is for this reason that we must look at the invariants in the natural order $E$, $A$, $\delta$, signatures.
\end{remark}

One may check that $G$ determines $E$, $A$, $\delta$ and the signatures uniquely when $n>1$. Indeed, we may check this locally.
When $G$ splits $E=K\times K$, when $E$ is quasi-split then $E$ is the unique quadratic extension over which $G$ splits.
For places of $K$ which split in $E$, $G$ is precisely the norm one elements of the division algebra $A$, this uniquely determines $A$.
For places which do not splite, $G$ is the special unitary group of a Hermitian form of discriminant $\delta$.
The Hermitian form preserved by a special unitary group is unique up to scaling, $\delta$ respectively the signatures are the invariants of such forms.

\begin{remark}
In the case of $n=1$, we may perform the same constructions, however the group does not uniquely determine invariants $E$, $A$, $\delta$, and signatures (if we fix $E$, the others are uniquely determined as above).
Indeed, there is no outer automorphism group, so the choice of $E$ is no longer an invariant.
Exceptional isomorphisms of groups tell us that for $n=1$ the group $G$ is the spin group of a rank $3$ quadratic space.
In odd rank the discriminant does not impact the isomorphism type of the orthogonal group.
The discriminant of the quadratic space does still play a role in defining the Clifford algebra, in the same way it plays a role in defining a unitary group. The Witt invariant of the quadratic space is still an invariant and determines the algebra $A$.

It is worth remembering that choosing an $E$ and a discriminant is equivalent to choosing $A$.

The following example illustrates that any two quadratic algebras have $2$-dimensional Hermitian spaces with isomorphic special unitary groups.
\begin{example}
Consider $\delta_1,\delta_2$ two elements of $K$.
Let $A$ be the quaternion algebra $(\delta_1,\delta_2)$.
Then $A$ is a $E_i=K(\sqrt{\delta_i})$ module for $i=1,2$ and a quadratic module over $K$.
Hermitian  polarization of the norm form relative to the module structures makes $A$ into Hermitian spaces over both $E_1$ and $E_2$.

In either case we have that $A^1(R) = \{ g\in (A\otimes_K R)^\times \mid N_{A/K} = 1 \}$ acts by left multiplication on $A$ and preserves the norm form and thus both Hermitian forms. By base changing to $\overline{K}$ we redily verify that $A^1 = SU(H_i)$ for $i=1,2$.

Over a local field the two isomorphism classes of $E_i$-Hermitian spaces arise from the two isomorphism classes of Quaternion algebra.
Consequently both $E_1$ special unitary groups are isomorphic to $E_2$ special unitary groups.
How isomorphism classes pair off is controlled by $(\delta_1,\delta_2)$.
\end{example}
\end{remark}




\subsection{Quadratic Spaces}
 

Fix a free $K$-module $V$ of dimension $n$. In characteristic not $2$ to give a quadratic module structure on $K$ is equivalent to giving a non-degenerate symetric $K$-valued $K$-bilinear pairing $B$ on $V$.

$O(B) = \Aut(V,B)$ and $SO(H) = \Aut(V,B,\det)$ where $\det$ refers to the $K$-module determinant on $V$.

The forms of $V$ are in bijection with $H^1(\GalKbK, \Aut(V,O))$.
We have an exact sequence:
\[ 1 \rightarrow SO(B) \rightarrow O(B) \rightarrow \mu_2  \rightarrow 1. \]
This induces an exact sequence:
\[ \ldots \rightarrow H^1(\GalKbK,SO(B))  \rightarrow H^1(\GalKbK,O(B))  \rightarrow H^1(\GalKbK,\mu_2 ) \rightarrow \ldots. \]
We have an isomorphism $H^1(\GalKbK, \{\pm1\}) = K^\times/(K^\times)^2$ and we shall refer to the associated invariant of $B$ as the discriminant.
We thus have that the cohomology set $H^1(\GalKbK,SO(B))$ classifies forms with the same discriminant as $B$.
The group $SO(B)$ has a simply connected cover $Spin(B)$, and there is an exact sequence:
\[ \ldots \rightarrow H^1(\GalKbK,Spin(B))  \rightarrow H^1(\GalKbK,SO(B))  \rightarrow H^2(\GalKbK \mu_2 ).\]
The group $H^2(\GalKbK,\{ \pm 1 \} )$ classifies quaternion algebras over $K$, we may thus associate to our form a quaternion algebra, which we shall refer to as the Hasse invariant.
The set $H^1(\GalKbK,Spin(B)) $ then parameterizes forms with the same discriminant and Hasse invariant. As the group is simply connected, such forms are controlled at the real places of $K$, they are parametrized by the signature $(r_\nu,s_\nu)$.

We summarize the above with the following well known proposition.
\begin{proposition}
Fix a free $K$-module $V$.
Quadratic structures on $V$ are classified up to isomorphism by their discriminants $\delta \in K^\times/(K^\times)^2$, Hasse invariants $[A] \in  H^2(\GalKbK,\{ \pm 1 \} )$ and their signatures $(s_\nu,r_\nu)$ at real places of $K$ which ramify in $E$.
Moreover, $\delta$ may be computed as the determinant of the Grahm matrix whereas $s_\nu$ and $r_\nu$ are respectively the number of positive and negative eigenvalues of the Grahm matrix in diagonal form.
The Hasse invariant may be computed as $\prod_{i<j}(e_i,e_j)$ where $e_i,e_j$ are the eigenvalues for a Grahm matrix in diagonal form.
\end{proposition}

\begin{remark}
There are several subtly different ways to define the Hasse (or Witt or Hasse-Witt) invariant, these typically differ by multiplication in the Brauer group by a constant depending on the discriminant.
The one we have given is not equal to the coboundary map for the exact sequence above.
\end{remark}


\subsection{Orthogonal Groups}

We shall be making use of orthogonal groups in even dimensions.
For the purpose of this paper an even dimensional orthogonal group means a non-simply connected, non-adjoint group of type $D_n$. When $n\neq 4$ is even we additionally require that the kernel of the map from the simply connected cover be the same as the kernel of its vector representation.

\begin{remark}
When $n=1$ this group is $PSL_2$, when $n=2$ this group $SL_2\times SL_2/\pm1$, when $n=3$ this group is $PSL_4$.
\end{remark}

\begin{remark}
When $n$ is even the center of the simply connected cover is $(\ZZ/2\ZZ)^2$ (possibly twisted), when it is odd it is $(\ZZ/4\ZZ)$ (possibly twisted), hence we only need the additional conditions clarifying which quotients we consider in teh case of $n$ even.
In any case, as our quotient group is rational, the kernel from the simply connected cover is $\ZZ/2\ZZ$ (not twisted).
\end{remark}

When $n>1$ the outer automorphism group of such a group has order $2$.
\begin{remark}
Note that the triality automorphisms of $D_4$ does not descend to this quotient as the triality automorpism acts as the automorphism group of $(\ZZ/2\ZZ)^2$ on the center of the spin group. Thus no outer twist with non-trivial action via a three cycle contains rational elements in the center.
\end{remark}

We may construct the orthogonal groups as defined above as follows. Fix the following:
\begin{itemize}
\item A quaternion algebra $A$ over $K$ (we allow $M_2(K)$).
\item Fix an orthogonal involution $\tau$ of $A$, for a quaternion algebra take $\tau(x) = i\sigma(x)i^{-1}$ for any $i\in A$ with $\sigma(i) = -i$, for $\sigma$ the standard involution of $A$, for the matrix algebra take $\tau$ to be the transpose.

(An involution is orthogonal if the space of elements $x\in A$ with $\tau(x) = x$ has dimension $n(n+1)/2$ over $K$, otherwise it is symplectic and this dimension is $n(n-1)/2$ see \cite[2.6]{TODO the book of involutions}).
\item An element $J \in M_{n}(A)^\times$ such that $J = \tau(J^t)$.
Note that choosing $J$ is equivalent to the choice of orthogonal involution $\tau$.
\end{itemize}
Define then $\tilde{G}$ to be the group scheme whose points over a $K$-algebra $R$ are given by:
\[ \tilde{G}(R) = \{ g\in (M_{n}(A) \otimes_K R)^\times \mid \tau(g^t)Jg = J  \}. \]
Its connected component is the group scheme:
\[ G(R) = \{ g\in (M_{n}(A) \otimes_K R)^\times \mid \tau(g^t)Jg = J  \text{ and } Nm(g) = 1 \}. \]
We shall denote the simply connected cover by $\hat{G}$.

We claim that all orthogonal groups arise this way (note that $\hat{G}$ admits further forms in the case $n=4$).

The outer automorphism group of such a group has order $2$.
It is generated by an element of $\tilde{G}\setminus G$.
We thus have an exact sequence:
\[ 1 \rightarrow \mu_2 \rightarrow \tilde{G} \rightarrow \Aut(G) \rightarrow 1 \]
from which we derive the exact sequence:
\[\ldots \rightarrow H^1(\GalKbK,\mu_2) \rightarrow  H^1(\GalKbK, \tilde{G})  \rightarrow  H^1(\GalKbK, \Aut(G)) \rightarrow  H^2(\GalKbK, \mu_2)  \]
We may thus associate to $G$ a quaternion algebra $A$, and an element of of the cohomology set $H^1(\GalKbK, \tilde{G})$ modulo the action of the cohomology group $H^1(\GalKbK,\mu_2)$.

By \cite[29.16]{TODO - The book of involutions}
the cohomology set $H^1(\Gamma, Isom(A,\tau)) = H^1(\Gamma,\tilde{G}) $ classifies the '$\tau$-symmetric elements of $A^\times$' up to change of basis.
The action of $H^1(\GalKbK,\mu_2) $ is to rescale these by an element of $k^\times/(k^\times)^2$.
\begin{remark}
When $A$ is split, this is nothing other than the classification of quadratic forms up to rescaling as above.
In even dimensions, this depends on the discriminant and the base change of the Hasse invariant to `discriminant' algebra.

When $A$ is a quaternion algebra, the analysis proceeds as above, and we may still associate to $G$ a discriminant, a quaternion algebra and signature information. 
As we shall not need it in the sequel, we omit the details, but remark that among other simplifications, $J$ can be taken to be diagonal, and the diagonal entries can be rescaled by $(K^\times)^2$
\end{remark}


\subsection{From Unitary Groups to Orthogonal Groups}

In many cases unitary groups can be embedded into orthogonal groups, we shall only be making explicit use of this in the case where groups are associated to Hermitian and symmetric forms, however we present the more general situation.

Fix the following data:
\begin{itemize}
\item An quadratic \'etale algebra $E$ over $K$ with involution $\sigma_E$.
\item A quaternion algebra $A_0$ over $K$ with standard involution $\sigma_{A_0}$, then define $A = A_0\otimes_K E$ with an involution of the second kind $\sigma_A = \sigma_{A_0} \otimes \sigma_E$.
(Note that all such $A$ arise uniquely this way see \cite[2.22]{TODO - book of involutions}).
\item An element $J\in F^\sigma$ of norm $\delta$ whose minimal polynomial at a real place $\nu$ of $K$ which ramifies in $E$ has $r_\nu$ positive roots and $s_\nu$ negative roots.
\end{itemize}
Then define then $SU$ to be the associated unitary group whose points over a $K$-algebra $R$ are given by:
\[ SU_J(R) = \{ g\in (A\otimes_K R)^\times \mid \sigma_A(g)Jg = J \text{ and } Nm(g) = 1 \}. \]

Now, fix a $K$-division algebra $B$ with a fixed embedding of $E$ into $B$. Denote the standard involution on $B$ by $\sigma_{B}$ and define an orthogonal involution by $\tau_B(x) = y\sigma(x)y^{-1}$, where $y\in E$ has $\sigma_B(y) = -y$.

Now define the algebra $C = A_0 \otimes_K B$, the involution $\sigma_C = \sigma_{A_0} \otimes \sigma_B$ is the standard involution and the involution $\tau_C = \sigma_{A_0} \otimes \tau_B$ is an orthogonal involution.
Both $\sigma_C$ and $\tau_C$ restrict to $\sigma_A$ on the embedded image of $A$ in $C$.

We may thus define the group schemes $SO_J$ and $SP$:
\[ SO_J(R) =  \{ g\in (M_{n}(C) \otimes_K R)^\times \mid \tau_C(g^t)Jg = J  \} \]
and
\[ SP(R) =  \{ g\in (M_{n}(C) \otimes_K R)^\times \mid \sigma_C(g^t)Jg = J  \}. \]
We redilly observe that $SU_J$ embedds naturally into both of these groups.
\begin{remark}
The group $SP$ is a symplectic group, its isomorphism class does not depend on $J$.
\end{remark}

\begin{remark}
In the special case where $A_0$ and $B$ are split, so that $SU_J$ is the special unitary group of an $E$-valued Hermitian form $H$ associated to $J$ then
$SO_J$ is the orthogonal group associated to the $K$-valued bilinear form $B(z_1,z_2) = \Tr_{E/K}(H(z_1,z_2))$ and the group $SP$ is the symplectic group associated to the symplectic form $S(z_1,z_2) = \Tr_{E/K}(H(yz_1,z_2))$.

Note that given $SO_J$ or $SP$, and the $E$-module structure on the standard representations, the group $SU_J$ may be recovered.
Note also that $U_J = SO_J \cap SP$.
\end{remark}










\section{\'Etale Algebras  associated to Tori in Algebra Groups}

Let $T$ be a torus defined over $K$, $V$ a vector space over $K$ and $\rho : T \rightarrow \GL(V)$ an inclusion of $K$-groups.  After diagonalizing the image of $\rho$ over $\Kb$, projecting onto each coordinate gives a finite set of characters of $T$; let $X$ denote the set of characters arising in this way. 

Associated to this data there exists an \'etale algebra $E$ over $K$, inclusions $\rho'' : E \hookrightarrow \End(V)$ and $\rho' : T \rightarrow T_E$ for which $\rho = \rho''\circ\rho'$.

\begin{proposition} \label{prop:etale-construction}
The following constructions produce isomorphic \'etale algebras $E$ over $K$:
\begin{enumerate}
\item \label{prop:etale-construction1} The set $X$ has an action of $\GalKbK$ and thus determines an \'etale algebra $E$.
\item \label{prop:etale-construction2} Let $C = C_{\End(V)}(\rho(T))$ be the centralizer of $\rho(T)$ in $\End(V)$ and $Z$ be its center, both as schemes over $K$.  Then $E = Z(K)$ is an \'etale algebra.
\item \label{prop:etale-construction3} If $K$ is infinite, then $E$ can be taken as the $K$-span in $\End(V)$ of $\rho(T(K))$.
\end{enumerate}
\end{proposition}
\begin{proof}
That \eqref{prop:etale-construction2} and \eqref{prop:etale-construction3} give \'etale algebras follows from base change to the algebraic closure.
That the algebra constructed by \eqref{prop:etale-construction3} is contained in that constructed by \eqref{prop:etale-construction2} is clear.
The converse can be checked over the algebraic closure, using flatness and dimension considerations.

The equivalence of \eqref{prop:etale-construction1} and \eqref{prop:etale-construction3} follows from the characterization of \'etale algebras by Galois descent.
\end{proof}

From description \eqref{prop:etale-construction2}, we obtain inclusions $\rho'' : E \hookrightarrow \End(V)$ and $\rho' : T \rightarrow T_E$ which satisfy $\rho = \rho''\circ\rho'$.  Furthermore, if $\dim_K E = \dim_K V$ then $\rho''$ gives $V$ the structure of a rank-$1$ $E$-module.

\subsection{Structures on $E$}

The choice of representation $\rho$ determines structures on $E$.

\begin{proposition}
Let $B : V_K \times V_K \rightarrow K$ be a non-degenerate bilinear form preserved by the image of $\rho$ then the adjoint involution on $\End(V_K)$ (relative to $B$) defines an involution $\sigma$ on $E$.
Moreover, $T \rightarrow T_{E\sigma} \rightarrow T_E$ is a factorization of the map $\rho'$.
\end{proposition}
\begin{proof}
Denote by $\sigma$ the adjoint involution on $\End(V_K)$.
We are given that $B(x,y) = B(\rho(t)x,\rho(t)y) = B(x,\sigma(\rho(t))\rho(t)y)$ for all $x$ and $y$. By the non-degeneracy of $B$ this implies $\sigma(\rho(t))\rho(t)$ is the identity and thus $\sigma(\rho(t)) = \rho(t)^{-1}$. Thus, $\sigma$ takes $T$ to $T$. by the naturality of the construction of $E$ above $\sigma$ takes $E$ to $E$.

It is now clear that $\rho(T) \subset T_{E,\sigma}$.
\end{proof}



\begin{proposition}
Let $A$ be an algebra over $K$ and suppose $V_K$ has the structure of an $A$-module.
Suppose $\rho(T)$ embeds into $\End_A(V_K)$ then $E\hookrightarrow \End_A(V_K)$.

Suppose $\rho(T)$ commutes with the action of $A$, then the image of $E$ commutes with the action of $A$.
If moreover the $K$-dimension of $E$ equals that of $V$ this also gives $E$ the structure of an $A$-algebra.
\end{proposition}
\begin{proof}
By naturality of construction if $\rho(T)$ acts by $A$ endomorphisms then so too does $E$.
Likewise if $\rho(T)$ commutes with the action of $A$, then so too does $E$.

Finally, if $K$-dimensions are equal, then by identifying $V_K$ with $E$ we have given $E$ the structure of an $A$-module. Since the $A$ and $E$ multiplication commute, we are giving $E$ the structure of an $A$-algebra.
\end{proof}
\begin{remark}
The case of non-commutative $E$ is mostly relevant to inner forms of symplectic groups.
In what follows we will assume $A$ is a commutative algebra over $K$.
\end{remark}

\begin{proposition}
Let $A$ be a quadratic algebra over $K$. Suppose $V_K$ has the structure of an $A$-module, and $B : V_K \times V_K \rightarrow A$ is a non-degenerate Hermitian form.
Suppose  $\rho(T)$ preserves $B$ then $E\hookrightarrow \End_A(V_K)$, the image of $E$ commutes with the action of $A$,  $E$ has the structure of an $A$-algebra, and the involution on $E$ restricts to that on $A$.
\end{proposition}
\begin{proof}
The only new statement here is that the involution on $E$ restricts to the involution on $A$.
The involution on $E$ was induced by the adjoint from $B$. But the involution on $A$ is by construction the adjoint.
\end{proof}
\begin{remark}
The $K$-algebra $E= A\times A$ has two meaningfully different $A$-algebra structures arising from the inclusions:
\[ a \mapsto (a,a) \qquad \text{and}\qquad a\mapsto (a,\sigma(a)) \]
once we impose an involution on $E$. The two cases are meaningfully distinguished by how the involution restricts to $A$.
\end{remark}


\subsection{Tori in Algebraic Groups}

Let $G$ be an algebraic group over $K$ and suppose $T \hookrightarrow G$.
Then any representation $\rho$ of $G$ restricts to a representation $\rho$ of $T$.

The structures imposed on $E_T$ by $\rho$ are determined more by the representation $\rho$ of $G$ and not directly by $T$. If $\rho$ is injective the consequence is that all tori (at least conjugate to $T$ over $\overline{K}$) in $G$ will have the same structures.

\begin{example}
Let $\rho:G \hookrightarrow \GL(V_K,B)$ be a representation preserving the bilinear form $B$ then for all tori $T$ in $G$, there exists an \'etale algebra with involution $(E,\sigma)$ and and inclusion $T \hookrightarrow T_{E,\sigma}$.

This holds even if $T$ is a split torus in the split unitary group $G=GL_n$ and $\rho$ is the representation corresponding to the standard representation of the unitary group. Ie, if the representation corresponds to the same vector in the $A_n$ root system.
\end{example}

\begin{example}
If the form $B$ above is Hermitian for an action of $A$ on $V_K$ then $E$ is an $A$-algebra with the involution on $E$ restricting to that on $A$.

In the split case the algebra $A$ is $K\oplus K$.
\end{example}

\section{Forms preserved by $T_{E,\sigma}$}

\subsection{Shape of the Forms}

\begin{theorem}
Suppose $B_1,B_2$ be bilinear (or Hermitian) forms on $E$ for which the adjoint maps $Ad_{B_1}$ and $Ad_{B_2}$ induce the same map $\sigma:E\rightarrow E$.
That is $B_i(ex,y) = B_i(x,\sigma(e)y)$, ie, we are in the case where $T_{E,\sigma}$ preserves $B_1$ and $B_2$.
Then, $B_1(x,y) = B_2(\lambda x,y)$.
\end{theorem}
\begin{proof}
Both $B_1$ and $B_2$ induce isomorphism between $E$ and its K (resp. A)-linear dual.
Hence, there exists a unique $\lambda\in E$ such that $B_1(1,y) = B_2(\lambda,y)$ for all $y\in E$.
We have:
\[ B_1(e,x) = B_1(1,\sigma(e)x) = B_2(\lambda,\sigma(e)x) = B_2(\lambda e,x). \]
\end{proof}
\begin{theorem}
If in the above if $B_1$ and $B_2$ are symmetric then $\sigma(\lambda)=\lambda$.

If in the above $B_1$ and $B_2$ are Hermitian then $\sigma(\lambda)=\lambda$.
\end{theorem}
\begin{proof}
We have the following calculation:
\[ B_2(\lambda y,x) = B_1(y,x) = B_1(x,y) = B_2(\lambda x,y) = B_2(x,\sigma(\lambda) y) = B_2(\sigma(\lambda)y,x) \]
Alternatively:
\[ B_2(\lambda y,x) = B_1(y,x) = \overline{B_1(x,y)} = B_2(\lambda x,y) = \overline{B_2(x,\sigma(\lambda) y)} = B_2(\sigma(\lambda)y,x) \]
And this holds for all $x,y$.
Hence, by the uniqueness of $\lambda$ from above, $\lambda=\sigma(\lambda)$.
\end{proof}
\begin{corollary}
All the non-degenerate symmetric bilinear forms on $E$ are of the form:
\[ Tr_{E/K}(\lambda x\sigma(y) \]
All the non-degenerate Hermitan forms on $E$ are of the form:
\[ Tr_{E/A}(\lambda x\sigma(y) \]
where $\lambda \in E^\sigma$ is a unit.
In both cases, the isomorphism class of the form depends only on $\lambda \in (E^\sigma)^\times/N_{E/E^\sigma}(E^\times)$.
\end{corollary}

\subsection{Invariants of Hermitian Forms}

\TODO - in this section we might benefit from renormalizing the form, in my experiance letting $\lambda' = \lambda/f'_z(z)$ where $z\in E^\sigma$ is such that $\sqrt{z}$ primitively generates $E$ over $K$ is a good choice.
(Such $z$ exist outside characteristic $2$ when $E$ is a field, or $E$ is an \'etale algebra and $K$ infinite).

\TODO - probably want a lemma about how, or perhaps explaining invariants of Hermitian forms.
$(d,\delta_{A/K}) $ for all intents and purposes is $d \in K^\times/N_{A/K}(A^\times)$.

\begin{notation}
Given an \'etale algebra with involution $(E,\sigma)$ of dimension $2n$ and $\lambda \in (E^\sigma)^\times$ let $z\in E^\sigma$ be such that $\sqrt{z}$ primitively generates $E$ over $K$ and denote by $f'_z(X)$ its minimal polynomial.
Define the Hermitian form:
\[ H_{E,\sigma,\lambda} := \tfrac{1}{2}Tr_{E/A}\left(\lambda x\sigma(y)\right) \]
and the symmetric bilinear form:
\[ B_{E,\sigma,\lambda} := \tfrac{1}{2}Tr_{E/K}\left(\lambda x\sigma(y)\right). \]
\end{notation}

\TODO - in all the following the Witt invariant formula might be nicer

\begin{proposition}\label{prop:Hasse}
The disciminant of $B_{E,\sigma,\lambda}$ is:
\[  (-1)^n\delta_{E/K}, \]
The Hasse invariant of $B_{E,\sigma,\lambda}$ is:
\[  \Cor_{E^\sigma/K}((\lambda (-1)^n f_z'(z), z)_{E^\sigma})\cdot(-1,-1)_K^{n(n-1)/2}.\] 
The signature of $B_{E,\sigma,\lambda}$ at a place where $E$ is complex and $E^\sigma$ is totally real is:
\[ 2(\#\lambda^+. \#\lambda^-) \]
is twice the number of positive and negative embeddings of $\lambda$.
\end{proposition}
This is \TODO{cite Thm 3.3, 3.8, 5.2 in my paper}.

\begin{proposition}\label{prop:InvariantRelations}
Let $H$ be any Hermitian form and denote:
\[ B(x,y) = Tr_{A/K}(H(x,y)). \]
Then:
\begin{enumerate}
\item The discriminant of $H$ (and the discriminant of $A$) determines the Hasse invariant of $B$. (The converse holds meaningfully whenever $A$ is not split, and is trivial otherwise.)
\item The discriminant of $A$ determines the discriminant of $B$.
\item The signatures of $H$ (and the infinite ramification of $A$) determines the signatures of $B$. (The converse holds meaningfully if $A$ is ramified, and is trivial otherwise.)
\end{enumerate}
In particular the invariants of $B$ are equivalent to those of $H$.
\end{proposition}
\begin{proof}
By using Proposition \ref{prop:Hasse}, and noting that $A = K(\sqrt{-\delta_{A/K}})$ one can compute that the Hasse invariant of the forms:

\TODO - notation for discriminants of hermitian forms
\[\Tr_{A/K}(dx_0\sigma(y_0)) \qquad \text{and} \qquad \Tr_{A/K}(x_1\sigma(y_1)) \]
are respectively:
\[ (-d \delta_{A/K}, \delta_{A/K})_K = (d,\delta_{A/K}) \qquad \text{and}\qquad ( -\delta_{A/K}, \delta_{A/K})_K = 1. \]
From this we quickly conclude that the Hasse invariant of:
\[ \Tr_{A/K}\left(dx_0\sigma(y_0) + \sum x_i\sigma(y_i)\right) \]
is precisely
\[ (d,\delta_{A/K})(-1,-1)_K^{n(n-1)/2}(\delta_{A/K},-1)_K^{n(n-1)/2}.\]
Conversely, it follows that:
\TODO - Notation
\[ (d,\delta_{A/K}) = ``Hasse''  (\delta_{A/K},-1)_K^{n(n-1)/2}(-1,-1)_K^{n(n-1)/2}. \]
At all places where $A/K$ is not split, this determines $d$ in $K^\times/N_{A/K}(A^\times)$.
At all places where $A/K$ is split, the norm map is surjective, and the discriminant is meaningless.
\TODO - phrasing that tells us we just determined $d$ globally.


By Proposition \ref{prop:Hasse} the discriminant of the form $B$ is precisely
\[ (-\delta_{A/K})^n. \]

For a real ramified place the signature of $B$ is precisely:
\[ (2s,2r) \]
where $(s,r)$ is the signature of $H$.
For a real split place the signature of $B$ is precisely:
\[ (n,n) \]
whereas, $H$ has no meaningful signature (the algebra $A$ is $\RR\times \RR$) as $-1$ is a norm.

For complex places there are no signatures in either case.
\end{proof}


\begin{proposition}\label{prop:discrimmap}
The discriminant of $H_{E,\sigma,\lambda}$ satisfies:
\[  (d,\delta_{A/K})_K =  \Cor_{E^\sigma/K}((\lambda (-1)^n f_z'(z), z)_{E^\sigma}) \cdot (\delta_{A/K},-1)_K^{n(n-1)/2}. \]
Moreover, the map:
\[ (E^\sigma)^\times/N_{E/E^\sigma}(E^\times) \rightarrow K^\times/N_{A/K}(A^\times) \]
given by:
\TODO{notation for hermitian discriminant}
\[ \lambda \mapsto (-1)^{n(n-1)/2} D\left( \tfrac{1}{2}\Tr_{E/A}\left( \frac{(-1)^n\lambda}{f'_z(z)} x\sigma(y) \right)\right) \]
is a homomorphism.
\end{proposition}
\begin{proof}
\TODO - complete sketch

Combining the results of Propositions \ref{prop:Hasse} and \ref{prop:InvariantRelations} we find that:
\TODO - details
\[ (d,\delta_{A/K}) =  \Cor_{E^\sigma/K}((\lambda,z)) \Cor_{E^\sigma/K}((-f'(z),z) (-1,-1)^{n(n-1)/2}...) ... \]
Now, if $\delta_{A/K}$ is a square then so is $z$

Provided $E/E^\sigma$ not split the formula should tell us non-trivial $d$ come from non-trivial $\lambda$.

Note that for a local field $(E^\sigma)^\times/N_{E/E^\sigma}(E^\times)$ is $( K^\times/N_{A/K}(A^\times))^\ell$ where $\ell$ is the number of field factors of $E$ which are $\sigma$-stable.
\end{proof}

\begin{proposition}\label{prop:discrimmapsurjective}
Let $K$ be $p$-adic fidle, $\RR$ or $\CC$ and fix $A$ a quadratic extension of $K$ (potentially split).
Fix $(E,\sigma)$ an \'etale algebra with involution (of the second kind) over $A$.
If $E$ is a field then the homomorphism of Proposition \ref{prop:discrimmap} is surjective.
\end{proposition}
\begin{proof}
The corestriction map is injective for local fields, and we are in the setting where both groups are isomorphic to $\{\pm1\}$, it follows immediately that the map is also surjective.
\end{proof}

\begin{proposition}\label{prop:signature}
Let $K$ be $\RR$ and $A$ be $\CC$.
If $E \simeq \CC$ then
The signature of $H_{E,\sigma,\lambda}$ is:
\[ (\#\lambda^+,\#\lambda^-) \]
If $E\simeq \CC\times\CC$ with $\sigma$ interchanging factors then the signature of $H_{E,\sigma,\lambda}$ is:
\[ (1,1). \]
\end{proposition}

\subsection{Existence of Forms}

\begin{theorem}
Let $K$ be a $p$-adic field and fix $A$ a quadratic extension of $K$ (potentially split).
Given $H$ a Hermitian form and $(E,\sigma)$ an \'etale algebra with involution (of the second kind) over $A$. Then there exists $\lambda \in E^\sigma$ with:
\[ H \simeq H_{E,\sigma,\lambda} \]
if and only if there exists a factor $E_i^\sigma$ of $E^\sigma$ over which the corresponding factor of $E_i$ is not split.
\end{theorem}
\begin{proof}
If $A$ is split over $K$ then there is a unique Hermitian space and hence nothing to show.

In this setting the cohomology groups and the corestriction map of Propostion \ref{prop:discrimmap} are products of those appearing in Proposition \ref{prop:discrimmapsurjective}. The map of Propostion \ref{prop:discrimmap} is thus surjective if the cohomology groups are non-trivial. This is the case precisely if at least one factor of $E_i$ over $E_i^\sigma$ is not split.
\end{proof}

\begin{example}
Fix a prime numbers $p_1,p_2$ and $p_3$ such that $\left(\tfrac{p_1}{p_3}\right) = 1$ and $\left(\tfrac{p_2}{p_3}\right) = -1$.
Set $A=\QQ(\sqrt{p_2})$ with the unique non-trivial involution, and $E=\QQ(\sqrt{p_2},\sqrt{p_1p_2})$ with the unique involution extending $\sigma$ acting trivially on $\sqrt{p_1p_2}$.
Then $A_{p_3} = \QQ_{p_3}(\sqrt{p_2})$ and $E_{p_3} \simeq A_{p_3} \times A_{p_3}$ with $\sigma$ interchanging factors.
It follows that the torus $T_{E,\sigma}$ does not embed into any unitary group which is not quasi-split at $p_3$.
\end{example}

\begin{theorem}
Let $K$ be $\RR$ or $\CC$ and fix $A$ a quadratic extension of $K$ (potentially split).
Given $H$ a Hermitian form and $(E,\sigma)$ an \'etale algebra with involution (of the second kind) over $A$. Then there exists $\lambda \in E^\sigma$ with:
\[ H \simeq H_{E,\sigma,\lambda} \]
if and only if 
\TODO - phrase signature conditions
\end{theorem}
\begin{proof}
If $A$ is split over $K$ then there is a unique Hermitian space and hence nothing to show.
This is the case unless $A=\CC$ and $K=\RR$. In this case $E$ is $\CC^n$, however some of the factors might be interchanged by $\sigma$.
By Proposition \ref{prop:signature} the factors interchanged by $\sigma$ must contribute $(1,1)$ to the signature.
Wheras those not interchanged contribute $(1,0)$ or $(0,1)$ depending on the choice of $\lambda$.
From this we conclude the result.
\end{proof}

\begin{example}
\TODO - A CM-field field $A$ and a complex field $E$ such that $E\otimes_\QQ \RR$ has factors interchanged by an involution.
\end{example}


\begin{lemma}\label{Lem:liftsigma}
Let Let $K$ be a number field and fix $A$ a quadratic extension of $K$ (potentially split).
Fix $(E,\sigma)$ an \'etale algebra with involution (of the second kind) over $A$.
Write $E = E_s \times E_{ns}$ where $E_s \simeq E_s^\sigma \times E_s^\sigma$ and $\sigma$ interchanges factors and $E_{ns}$ is a product of $\sigma$-stable fields.
There exists $\tilde{\sigma} \in \Gal(\overline{K}/K)$ inducing $\sigma$ on $E_{ns}$ (that is a unique $\tilde{\sigma}$ induces the involution on each factor.
\end{lemma}
\begin{proof}
Write:
\[ E_{ns} = \prod E_i \]
and set $\tilde{E} = E_1^\sigma E_2^\sigma\ldots E_{n}^\sigma$ to be the composite field of all the fixed fields of the $E_i$.
Notice that $E_i \simeq E_i^\sigma\otimes_K A \simeq E_i^\sigma A$.
Now, $A$ and $\tilde{E}$ are algebraicly disjoint fields and for all $i$ there exists canonical embeddings of fields:
\[ \xymatrix{ E_i \simeq E_i^\sigma\otimes_K A \ar@{^(->}[r]\ar@{-}[d] &  \tilde{E} \otimes_K A \simeq \tilde{E}A \ar@{-}[d] \\ 
                      E_i^\sigma \ar@{^(->}[r]\ar@{-} & \tilde{E}.}
 \]
The field $\tilde{E}A$ has a unique involution $\tilde{\sigma}$ such that $(\tilde{E}A)^{\tilde{\sigma}} = \tilde{E}$. Since $E_i^{\tilde{\sigma}} = E_i^\sigma$ this involution restricts to $\sigma$ on the image of $E_i$.
Finally, $\tilde{\sigma}$ lifts to an element of $\Gal(\overline{K}/K)$, giving us the claim.
\end{proof}

\begin{theorem}
Let $K$ be a number field and fix $A$ a quadratic extension of $K$ (potentially split).
Given $H$ a Hermitian form and $(E,\sigma)$ an \'etale algebra with involution (of the second kind) over $A$. Then there exists $\lambda \in E^\sigma$ with:
\[ H \simeq H_{E,\sigma,\lambda} \]
if and only if there exists $\lambda_\fp$ for all localizations $\fp$ of $K$ (both archimedian and non-archimedian).
\end{theorem}
\begin{proof}
\TODO - complete sketch (also check that this is true).
\TODO - need to distinguish primes of $K$, $A$ and $E$.

Write $E^\sigma = \prod_{i=1}^n E^\sigma_i$.
We have the following diagram:
\[
\xymatrix{
(E^\sigma)^\times/N_{E/E^\sigma}(E^\times) \ar[r]^{(\cdot,z)} \ar[d]& H^2(E^\sigma, \pm 1) \ar[r] \ar[d]& \underset{E_{i,\fp} \text{ not split}}\oplus H^2(E^\sigma_{i,\fp},\pm 1) \ar[d]\\
K^\times/N_{A/K}(A^\times) \ar[r]^{(\cdot,\delta_{A/K})} & H^2(K, \pm 1) \ar[r] & \underset{A_\fp \text{ not split}}\oplus H^2(K_\fp,\pm 1)
}
\]
where the compositions of the horizontal maps are injective.
We must show that an element in $K^\times/N_{A/K}(A^\times) $ is in the image of the left verticle map, provided its image is in the image of the right hand verticle map.
We must do this in spite of the fact that the horizontal maps are not surjective.

The failure of surjectivity is captured by the following, the bottom map surjects onto the kernel of the map:
\[ \underset{A_\fp \text{ not split}}\oplus H^2(K_\fp,\pm 1)  \rightarrow \{ \pm1\} \]
given by $(x_\fp) \mapsto \prod x_\fp$. The top map surjects onto the kernel of the map:
\[ \underset{E_{i,\fp} \text{ not split}}\oplus H^2(E_{i,\fp}^\sigma,\pm 1)  \rightarrow \{ \pm1\}^n \]
given by $(x_{i,\fp}) \mapsto (\prod x_{1,\fp}, \ldots, \prod x_{n,\fp}) $.

To complete the argument, it suffices to show that the kernel of the map:
\[\underset{E_{i,\fp} \text{ not split}}\oplus H^2(E_{i,\fp}^\sigma,\pm 1)  \rightarrow \underset{A_\fp \text{ not split}}\oplus H^2(K_\fp,\pm 1) \]
surjects onto the kernel of the map:
\[ \{ \pm1\}^n \rightarrow \{ \pm1\}. \]
To show this, it suffices to do this for a single fixed $\fp$ a prime of $K$.
Fix $\fp$ any prime of $K$ where $\tilde{\sigma}$ of Lemma \ref{Lem:liftsigma} acts as Frobenius. Then for all $i$ there exists $\fp_i | \fp$ prime of $E_i^\sigma$ for which $E_{\fp_i,i}$ is non-split. It follows that $\oplus_i H^2(E_{\fp_i,i}^\sigma,\pm1)$ surjects onto $\{\pm 1\}^n$, thus this also holds for the larger group, and hence also for the kernels.
\end{proof}

\begin{theorem}
\TODO - summarize a pure global existance statement.
\end{theorem}

\begin{remark}
\TODO - Can phrase this as: The reflex algebra splits the even clifford algebra + signature conditions.
Is there a meaningfull interpretation of the clifford algebra in this context?
Can we interpret it as telling us about a quaternioninc symplectic group embedding of the unitary group?
\end{remark}




\section{Properties of $T_{E,\sigma}$}

\subsection{Anisotropic}

\begin{proposition}
The torus $T_{E,\sigma}$ is anisotropic if all the field factors of $E$ are $\sigma$-stable.
\end{proposition}
\begin{proof}
\TODO-
That the condition is necissary is reasonably clear, sufficiency is less obvious.
\end{proof}

\begin{example}
Consider $E = A \times A$ where $\sigma$ acts on each factor.
This has the structure of an $A$ algebra, where $\sigma$ restricts to $\sigma$, by the inclusion $A\hookrightarrow A\times A$:
\[ a \mapsto (a,a) \]
The torus $T_{E,\sigma}$ is anisotropic.

\bigskip
Consider $E = A \times A$ where $\sigma$ acts to interchange factors.
Give this the structure of an $A$ algebra, where $\sigma$ restricts to $\sigma$, by the inclusion $A\hookrightarrow A\times A$:
\[ a \mapsto (a,\sigma(a)) \]
The torus $T_{E,\sigma}$ is not anisotropic.
\end{example}


\subsection{Unramified}

\begin{proposition}
The torus $T_{E,\sigma}$ is unramified if all the field factors of $E$ are unramified
\end{proposition}
\begin{proof}
\TODO-
\end{proof}


\subsection{``Relatively'' Unramified}

Your special unramifiedness when $A/K$ was ramified.

\section{The Building of the Unitary Group}

\subsection{The reduction of $T_{E,\sigma}$}

\section{Cohomological Reinterpretation}

This is stuff I don't recall super well, I think the following is roughly true, you might recall better than I.

\begin{question}
The set of pairs $(T,U)$ consisting of $T$ a maximal torus (up to rational conjugacy) of $U$ and $U$ a pure inner form of a unitary group are in bijection with:
\[ H^1( \Gal(\overline{K}/K), N_{U_0}(T_0) ) \]
\end{question}
If $T_0$ is split this becomes:
\[ H^1( \Gal(\overline{K}/K), W(U,T_0) ) \]

\begin{question}
The set of pure inner forms of $U$ containing a torus isomorphic to $T_0$ is in bijection with the kernel of the natural map:
\[ H^1( \Gal(\overline{K}/K), N_{U_0}(T_0) ) \rightarrow H^1( \Gal(\overline{K}/K),N_{GL_n}(T_E )) \]
\end{question}

\begin{question}
the set of maximal tori $T$ in $U_0$ (up to rational conjugacy) is in bijection with the kernel of the natural map:
\[  H^1( \Gal(\overline{K}/K), N_{U_0}(T_0) ) \rightarrow H^1( \Gal(\overline{K}/K), U_0) \]
\end{question}

The above should somehow give us that:
\begin{question}
The forms of $T$ contained in $U$ are in bijection with:
\[ H^1(\Gal(\overline{K}/K), W(U_0,T_0)^{\Gal(\overline{K}/K} ) \]
\end{question}
However this could plausibly require that $T_0$ be split or that $U_0$ is quasi-split.










\end{document}