\documentclass{article}
\usepackage{amsmath,amscd,amssymb,latexsym, amsfonts}
%\usepackage{mathtools}
\usepackage{amsthm}
\usepackage{xypic}
\usepackage{yfonts, mathrsfs}

\theoremstyle{plain}
\newtheorem{iassumption}{Assumption}
\newtheorem{theorem}{Theorem}[section]
\newtheorem{conjecture}[theorem]{Conjecture}
\newtheorem{proposition}[theorem]{Proposition}
\newtheorem{corollary}[theorem]{Corollary}
\newtheorem{hypothesis}[theorem]{Hypothesis}
\newtheorem{assumption}[theorem]{Assumption}
\newtheorem{lemma}[theorem]{Lemma}
\newtheorem{question}[theorem]{Question}
\newtheorem{exercise}[theorem]{Exercise}
\newtheorem{statement}[theorem]{Statement}
\newtheorem{example}[theorem]{Example}

\theoremstyle{definition}
\newtheorem{definition}[theorem]{Definition}
\newtheorem{notation}[theorem]{Notation}
\newtheorem{remark}[theorem]{Remark}

\DeclareMathOperator{\Gal}{Gal}
\DeclareMathOperator{\val}{val}
\DeclareMathOperator{\HH}{H}
\DeclareMathOperator{\Ad}{Ad}
\DeclareMathOperator{\Nm}{Nm}
\DeclareMathOperator{\Hom}{Hom}
\DeclareMathOperator{\Spec}{Spec}
\DeclareMathOperator{\Res}{Res}
\DeclareMathOperator{\Fr}{Fr}
\DeclareMathOperator{\Tr}{Tr}
\DeclareMathOperator{\Ind}{Ind}
\DeclareMathOperator{\gen}{gen}
\DeclareMathOperator{\End}{End}
\DeclareMathOperator{\Cor}{Cor}

\DeclareMathOperator{\GL}{GL}
\DeclareMathOperator{\PGL}{PGL}
\DeclareMathOperator{\SL}{SL}

\newcommand{\mat}[4]{\left( \begin{array}{cc} {#1} & {#2} \\ {#3} & {#4}
\end{array} \right)}
\newcommand{\TT}{\mathcal{T}}
\newcommand{\C}{\mathcal{C}}
\newcommand{\RR}{\mathbb{R}}
\newcommand{\CC}{\mathbb{C}}
\newcommand{\CCx}{\mathbb{C}^\times}
\newcommand{\OK}{\mathcal{O}_K}
\newcommand{\OKn}{\mathcal{O}_{K_n}}
\newcommand{\pK}{\mathfrak{p}_K}
\newcommand{\pL}{\mathfrak{p}_L}
\newcommand{\OL}{\mathcal{O}_L}
\newcommand{\ZZ}{\mathbb{Z}}
\newcommand{\QQ}{\mathbb{Q}}
\newcommand{\Qp}{\mathbb{Q}_p}
\newcommand{\Gm}{\mathbb{G}_m}
\newcommand{\Lx}{L^\times}
\newcommand{\fp}{\mathfrak{p}}

\newcommand{\Weil}{\mathcal{W}}
\newcommand{\WD}{\mathcal{W}'}
\newcommand{\Lpack}{\mathcal{L}}
\newcommand{\Pgen}{P_G^{\gen}}
\newcommand{\bmu}{\boldsymbol\mu}
\newcommand{\mugen}{\bmu^{\gen}}

\newcommand{\la}{\langle}
\newcommand{\ra}{\rangle}

\newcommand{\invlim}[1]{\varprojlim_{#1}}
\newcommand{\Normalizer}[2]{\operatorname{N}_{#2}(#1)}

\newcommand{\TODO}[1]{\textbf{TODO-#1}}

\newcommand{\aK}{K}
\newcommand{\gK}{\mathcal{K}}
\newcommand{\lK}{\mathfrak{K}}
\newcommand{\fK}{\textswab{K}}
\newcommand{\aL}{L}
\newcommand{\gL}{\mathcal{L}}
\newcommand{\lL}{\mathfrak{L}}
\newcommand{\fL}{\textswab{L}}
\newcommand{\aE}{E}
\newcommand{\gE}{\mathcal{E}}
\newcommand{\lE}{\mathfrak{E}}
\newcommand{\fE}{\textswab{E}}
\newcommand{\aV}{V}
\newcommand{\gV}{\mathcal{V}}
\newcommand{\lV}{\mathfrak{V}}
\newcommand{\fV}{\textswab{V}}

\title{Algebraic Tori in Unitary Groups}
\author{Andrew Fiori \& David Roe}

\begin{document}

\maketitle

\begin{abstract}
We give a characterization of the algebraic tori in unitary groups in terms of the structure of \`etale algebras with involution. We determine relavent properties, such as unramifiedness or anisotropicness, of the tori from the associated algebra. We discuss the implications on the structure of the building for the unitary group. We give a cohomological reinterpretation of the results.
\end{abstract}

\tableofcontents

\section{Introduction}

The problem of giving a classification and/or characterization of algebraic tori in algebraic groups has several sources of motivation coming out of algebraic number theory. In particular it has applications in the the theory of automorphic forms, Shimura varieties and the Langlands conjectures. Moreover, the algebraic tori play a key role in the construction of the building associated to the group, and also in the classification of forms of algebraic groups. Beyond all this the question is natural from a theoretical point of view in the theory of algebraic groups.  

\TODO{More Motivational nonsense}

\TODO{References to important literature}

\TODO{Summary of Layout}

\section{Definitions, Notation and Background}

\TODO{} Decide which of the following to define/give notations for, we probably don't need all of it. Some of it can be postponed until the sections where it is used.
\begin{itemize}
\item Explain our notation for fields and \'etale algebras: $\aK$, $\gK$, $\lK$, $\fK$, $\aL$, $\gL$, $\lL$, $\fL$, $\aE$, $\gE$, $\lE$, $\fE$, $\aV$, $\gV$, $\lV$, $\fV$.
\item The setting is a base field $K$ taken to be a $p$-adic field or number field (should we notationally distinguish? I don't currently).
\item We shall denote by $...$ the absolute Galois group of $K$.
\item \'Etale algebras are... typically denoted by $E$.
\item \'Etale algebras with involutions are...
\item Algebraic Tori are... 
\item Hermitian Spaces are ... their invariants over (local fields, global fields) are...
\item Unitary groups are ... their invariants over (local fields, global fields) are...
\item Quadratic Forms are ...their invariants over (local fields, global fields) are...
\item Orthogonal Groups are ...their invariants over (local fields, global fields) are...
\item Connections between orthogonal and unitary groups.
\end{itemize}


\section{Tori and \'Etale Algebras}

Let $T$ be a torus defined over $\aK$, $\aV$ a vector space over $\aK$ and $\rho:T_\aK \rightarrow \GL(V)$ an injective $\aK$-rational algebraic representation of $T$.

Associated to this data there exists an \'etale algebra $\aE$ over $\aK$, inclusions $\rho'' : \aE \hookrightarrow \End(V_K)$ and $\rho' : T \rightarrow T_E$ for which $\rho = \rho''\circ\rho'$.

\begin{proposition}
We have the following constructions of $E$:
\begin{enumerate}
\item Let $X$ denote the set of characters of $T$ appearing via the representation of $\rho$ over $\overline{K}$. This is a Galois set for the action of $\Gal(\overline{K}/K)$ hence defines a unique \'etale algebra $E$.
\item Let $C = C_\rho(T)$ be the centralizer of the image of $\rho(T)$ in $\End(V_K)$ (as a scheme over $K$).
           Let $Z = Z(C_\rho(T))$ be the centre of $C$ (as a scheme over $K$).
           Let $E = Z(K)$.
\item If the characters $\chi$ appearing in $X$ are distinguished by their values on $K$ points then $E$ is the $K$-span in $\End(V_K)$ of $\rho(T(K))$.
\end{enumerate}
\end{proposition}
\begin{proof}
That (2) and (3) give \'etale algebras follows from base change to the algebraic closure.
That the algebra constructed by (3) is contained in that constructed by (2) is clear.
The converse can be checked over the algebraic closure, using flatness and dimension considerations.

The equivalence of (1) and (3) follows from the characterization of \'etale algebras by Galois descent.
\end{proof}

\begin{remark}
If the size of $X$ equivalently the $K$-dimension of $E$ equals the $K$ dimension of $V_K$ then $V_K$ is a rank $1$ module over $E$ hence may be (non-canonically) identified with $E$.
\end{remark}

\subsection{Structures on $E$}

The choice of representation $\rho$ determines structures on $E$.

\begin{proposition}
Let $B : V_K \times V_K \rightarrow K$ be a non-degenerate bilinear form preserved by the image of $\rho$ then the adjoint involution on $\End(V_K)$ (relative to $B$) defines an involution $\sigma$ on $E$.
Moreover, $T \rightarrow T_{E\sigma} \rightarrow T_E$ is a factorization of the map $\rho'$.
\end{proposition}
\begin{proof}
Denote by $\sigma$ the adjoint involution on $\End(V_K)$.
We are given that $B(x,y) = B(\rho(t)x,\rho(t)y) = B(x,\sigma(\rho(t))\rho(t)y)$ for all $x$ and $y$. By the non-degeneracy of $B$ this implies $\sigma(\rho(t))\rho(t)$ is the identity and thus $\sigma(\rho(t)) = \rho(t)^{-1}$. Thus, $\sigma$ takes $T$ to $T$. by the naturality of the construction of $E$ above $\sigma$ takes $E$ to $E$.

It is now clear that $\rho(T) \subset T_{E,\sigma}$.
\end{proof}



\begin{proposition}
Let $A$ be an algebra over $K$ and suppose $V_K$ has the structure of an $A$-module.
Suppose $\rho(T)$ embeds into $\End_A(V_K)$ then $E\hookrightarrow \End_A(V_K)$.

Suppose $\rho(T)$ commutes with the action of $A$, then the image of $E$ commutes with the action of $A$.
If moreover the $K$-dimension of $E$ equals that of $V$ this also gives $E$ the structure of an $A$-algebra.
\end{proposition}
\begin{proof}
By naturality of construction if $\rho(T)$ acts by $A$ endomorphisms then so too does $E$.
Likewise if $\rho(T)$ commutes with the action of $A$, then so too does $E$.

Finally, if $K$-dimensions are equal, then by identifying $V_K$ with $E$ we have given $E$ the structure of an $A$-module. Since the $A$ and $E$ multiplication commute, we are giving $E$ the structure of an $A$-algebra.
\end{proof}
\begin{remark}
The case of non-commutative $E$ is mostly relevant to inner forms of symplectic groups.
In what follows we will assume $A$ is a commutative algebra over $K$.
\end{remark}

\begin{proposition}
Let $A$ be a quadratic algebra over $K$. Suppose $V_K$ has the structure of an $A$-module, and $B : V_K \times V_K \rightarrow A$ is a non-degenerate Hermitian form.
Suppose  $\rho(T)$ preserves $B$ then $E\hookrightarrow \End_A(V_K)$, the image of $E$ commutes with the action of $A$,  $E$ has the structure of an $A$-algebra, and the involution on $E$ restricts to that on $A$.
\end{proposition}
\begin{proof}
The only new statement here is that the involution on $E$ restricts to the involution on $A$.
The involution on $E$ was induced by the adjoint from $B$. But the involution on $A$ is by construction the adjoint.
\end{proof}
\begin{remark}
The $K$-algebra $E= A\times A$ has two meaningfully different $A$-algebra structures arising from the inclusions:
\[ a \mapsto (a,a) \qquad \text{and}\qquad a\mapsto (a,\sigma(a)) \]
once we impose an involution on $E$. The two cases are meaningfully distinguished by how the involution restricts to $A$.
\end{remark}


\subsection{Tori in Algebraic Groups}

Let $G$ be an algebraic group over $K$ and suppose $T \hookrightarrow G$.
Then any representation $\rho$ of $G$ restricts to a representation $\rho$ of $T$.

The structures imposed on $E_T$ by $\rho$ are determined more by the representation $\rho$ of $G$ and not directly by $T$. If $\rho$ is injective the consequence is that all tori (at least conjugate to $T$ over $\overline{K}$) in $G$ will have the same structures.

\begin{example}
Let $\rho:G \hookrightarrow \GL(V_K,B)$ be a representation preserving the bilinear form $B$ then for all tori $T$ in $G$, there exists an \'etale algebra with involution $(E,\sigma)$ and and inclusion $T \hookrightarrow T_{E,\sigma}$.

This holds even if $T$ is a split torus in the split unitary group $G=GL_n$ and $\rho$ is the representation corresponding to the standard representation of the unitary group. Ie, if the representation corresponds to the same vector in the $A_n$ root system.
\end{example}

\begin{example}
If the form $B$ above is Hermitian for an action of $A$ on $V_K$ then $E$ is an $A$-algebra with the involution on $E$ restricting to that on $A$.

In the split case the algebra $A$ is $K\oplus K$.
\end{example}

\section{Forms preserved by $T_{E,\sigma}$}

\subsection{Shape of the Forms}

\begin{theorem}
Suppose $B_1,B_2$ be bilinear (or Hermitian) forms on $E$ for which the adjoint maps $Ad_{B_1}$ and $Ad_{B_2}$ induce the same map $\sigma:E\rightarrow E$.
That is $B_i(ex,y) = B_i(x,\sigma(e)y)$, ie, we are in the case where $T_{E,\sigma}$ preserves $B_1$ and $B_2$.
Then, $B_1(x,y) = B_2(\lambda x,y)$.
\end{theorem}
\begin{proof}
Both $B_1$ and $B_2$ induce isomorphism between $E$ and its K (resp. A)-linear dual.
Hence, there exists a unique $\lambda\in E$ such that $B_1(1,y) = B_2(\lambda,y)$ for all $y\in E$.
We have:
\[ B_1(e,x) = B_1(1,\sigma(e)x) = B_2(\lambda,\sigma(e)x) = B_2(\lambda e,x). \]
\end{proof}
\begin{theorem}
If in the above if $B_1$ and $B_2$ are symmetric then $\sigma(\lambda)=\lambda$.

If in the above $B_1$ and $B_2$ are Hermitian then $\sigma(\lambda)=\lambda$.
\end{theorem}
\begin{proof}
We have the following calculation:
\[ B_2(\lambda y,x) = B_1(y,x) = B_1(x,y) = B_2(\lambda x,y) = B_2(x,\sigma(\lambda) y) = B_2(\sigma(\lambda)y,x) \]
Alternatively:
\[ B_2(\lambda y,x) = B_1(y,x) = \overline{B_1(x,y)} = B_2(\lambda x,y) = \overline{B_2(x,\sigma(\lambda) y)} = B_2(\sigma(\lambda)y,x) \]
And this holds for all $x,y$.
Hence, by the uniqueness of $\lambda$ from above, $\lambda=\sigma(\lambda)$.
\end{proof}
\begin{corollary}
All the non-degenerate symmetric bilinear forms on $E$ are of the form:
\[ Tr_{E/K}(\lambda x\sigma(y) \]
All the non-degenerate Hermitan forms on $E$ are of the form:
\[ Tr_{E/A}(\lambda x\sigma(y) \]
where $\lambda \in E^\sigma$ is a unit.
In both cases, the isomorphism class of the form depends only on $\lambda \in (E^\sigma)^\times/N_{E/E^\sigma}(E^\times)$.
\end{corollary}

\subsection{Invariants of Hermitian Forms}

\TODO - in this section we might benefit from renormalizing the form, in my experiance letting $\lambda' = \lambda/f'_z(z)$ where $z\in E^\sigma$ is such that $\sqrt{z}$ primitively generates $E$ over $K$ is a good choice.
(Such $z$ exist outside characteristic $2$ when $E$ is a field, or $E$ is an \'etale algebra and $K$ infinite).

\TODO - probably want a lemma about how, or perhaps explaining invariants of Hermitian forms.
$(d,\delta_{A/K}) $ for all intents and purposes is $d \in K^\times/N_{A/K}(A^\times)$.

\begin{notation}
Given an \'etale algebra with involution $(E,\sigma)$ of dimension $2n$ and $\lambda \in (E^\sigma)^\times$ let $z\in E^\sigma$ be such that $\sqrt{z}$ primitively generates $E$ over $K$ and denote by $f'_z(X)$ its minimal polynomial.
Define the Hermitian form:
\[ H_{E,\sigma,\lambda} := \tfrac{1}{2}Tr_{E/A}\left(\lambda x\sigma(y)\right) \]
and the symmetric bilinear form:
\[ B_{E,\sigma,\lambda} := \tfrac{1}{2}Tr_{E/K}\left(\lambda x\sigma(y)\right). \]
\end{notation}

\TODO - in all the following the Witt invariant formula might be nicer

\begin{proposition}\label{prop:Hasse}
The disciminant of $B_{E,\sigma,\lambda}$ is:
\[  (-1)^n\delta_{E/K}, \]
The Hasse invariant of $B_{E,\sigma,\lambda}$ is:
\[  \Cor_{E^\sigma/K}((\lambda (-1)^n f_z'(z), z)_{E^\sigma})\cdot(-1,-1)_K^{n(n-1)/2}.\] 
The signature of $B_{E,\sigma,\lambda}$ at a place where $E$ is complex and $E^\sigma$ is totally real is:
\[ 2(\#\lambda^+. \#\lambda^-) \]
is twice the number of positive and negative embeddings of $\lambda$.
\end{proposition}
This is \TODO{cite Thm 3.3, 3.8, 5.2 in my paper}.

\begin{proposition}\label{prop:InvariantRelations}
Let $H$ be any Hermitian form and denote:
\[ B(x,y) = Tr_{A/K}(H(x,y)). \]
Then:
\begin{enumerate}
\item The discriminant of $H$ (and the discriminant of $A$) determines the Hasse invariant of $B$. (The converse holds meaningfully whenever $A$ is not split, and is trivial otherwise.)
\item The discriminant of $A$ determines the discriminant of $B$.
\item The signatures of $H$ (and the infinite ramification of $A$) determines the signatures of $B$. (The converse holds meaningfully if $A$ is ramified, and is trivial otherwise.)
\end{enumerate}
In particular the invariants of $B$ are equivalent to those of $H$.
\end{proposition}
\begin{proof}
By using Proposition \ref{prop:Hasse}, and noting that $A = K(\sqrt{-\delta_{A/K}})$ one can compute that the Hasse invariant of the forms:

\TODO - notation for discriminants of hermitian forms
\[\Tr_{A/K}(dx_0\sigma(y_0)) \qquad \text{and} \qquad \Tr_{A/K}(x_1\sigma(y_1)) \]
are respectively:
\[ (-d \delta_{A/K}, \delta_{A/K})_K = (d,\delta_{A/K}) \qquad \text{and}\qquad ( -\delta_{A/K}, \delta_{A/K})_K = 1. \]
From this we quickly conclude that the Hasse invariant of:
\[ \Tr_{A/K}\left(dx_0\sigma(y_0) + \sum x_i\sigma(y_i)\right) \]
is precisely
\[ (d,\delta_{A/K})(-1,-1)_K^{n(n-1)/2}(\delta_{A/K},-1)_K^{n(n-1)/2}.\]
Conversely, it follows that:
\TODO - Notation
\[ (d,\delta_{A/K}) = ``Hasse''  (\delta_{A/K},-1)_K^{n(n-1)/2}(-1,-1)_K^{n(n-1)/2}. \]
At all places where $A/K$ is not split, this determines $d$ in $K^\times/N_{A/K}(A^\times)$.
At all places where $A/K$ is split, the norm map is surjective, and the discriminant is meaningless.
\TODO - phrasing that tells us we just determined $d$ globally.


By Proposition \ref{prop:Hasse} the discriminant of the form $B$ is precisely
\[ (-\delta_{A/K})^n. \]

For a real ramified place the signature of $B$ is precisely:
\[ (2s,2r) \]
where $(s,r)$ is the signature of $H$.
For a real split place the signature of $B$ is precisely:
\[ (n,n) \]
whereas, $H$ has no meaningful signature (the algebra $A$ is $\RR\times \RR$) as $-1$ is a norm.

For complex places there are no signatures in either case.
\end{proof}


\begin{proposition}\label{prop:discrimmap}
The discriminant of $H_{E,\sigma,\lambda}$ satisfies:
\[  (d,\delta_{A/K})_K =  \Cor_{E^\sigma/K}((\lambda (-1)^n f_z'(z), z)_{E^\sigma}) \cdot (\delta_{A/K},-1)_K^{n(n-1)/2}. \]
Moreover, the map:
\[ (E^\sigma)^\times/N_{E/E^\sigma}(E^\times) \rightarrow K^\times/N_{A/K}(A^\times) \]
given by:
\TODO{notation for hermitian discriminant}
\[ \lambda \mapsto (-1)^{n(n-1)/2} D\left( \tfrac{1}{2}\Tr_{E/A}\left( \frac{(-1)^n\lambda}{f'_z(z)} x\sigma(y) \right)\right) \]
is a homomorphism.
\end{proposition}
\begin{proof}
\TODO - complete sketch

Combining the results of Propositions \ref{prop:Hasse} and \ref{prop:InvariantRelations} we find that:
\TODO - details
\[ (d,\delta_{A/K}) =  \Cor_{E^\sigma/K}((\lambda,z)) \Cor_{E^\sigma/K}((-f'(z),z) (-1,-1)^{n(n-1)/2}...) ... \]
Now, if $\delta_{A/K}$ is a square then so is $z$

Provided $E/E^\sigma$ not split the formula should tell us non-trivial $d$ come from non-trivial $\lambda$.

Note that for a local field $(E^\sigma)^\times/N_{E/E^\sigma}(E^\times)$ is $( K^\times/N_{A/K}(A^\times))^\ell$ where $\ell$ is the number of field factors of $E$ which are $\sigma$-stable.
\end{proof}

\begin{proposition}\label{prop:discrimmapsurjective}
Let $K$ be $p$-adic fidle, $\RR$ or $\CC$ and fix $A$ a quadratic extension of $K$ (potentially split).
Fix $(E,\sigma)$ an \'etale algebra with involution (of the second kind) over $A$.
If $E$ is a field then the homomorphism of Proposition \ref{prop:discrimmap} is surjective.
\end{proposition}
\begin{proof}
The corestriction map is injective for local fields, and we are in the setting where both groups are isomorphic to $\{\pm1\}$, it follows immediately that the map is also surjective.
\end{proof}

\begin{proposition}\label{prop:signature}
Let $K$ be $\RR$ and $A$ be $\CC$.
If $E \simeq \CC$ then
The signature of $H_{E,\sigma,\lambda}$ is:
\[ (\#\lambda^+,\#\lambda^-) \]
If $E\simeq \CC\times\CC$ with $\sigma$ interchanging factors then the signature of $H_{E,\sigma,\lambda}$ is:
\[ (1,1). \]
\end{proposition}

\subsection{Existence of Forms}

\begin{theorem}
Let $K$ be a $p$-adic field and fix $A$ a quadratic extension of $K$ (potentially split).
Given $H$ a Hermitian form and $(E,\sigma)$ an \'etale algebra with involution (of the second kind) over $A$. Then there exists $\lambda \in E^\sigma$ with:
\[ H \simeq H_{E,\sigma,\lambda} \]
if and only if there exists a factor $E_i^\sigma$ of $E^\sigma$ over which the corresponding factor of $E_i$ is not split.
\end{theorem}
\begin{proof}
If $A$ is split over $K$ then there is a unique Hermitian space and hence nothing to show.

In this setting the cohomology groups and the corestriction map of Propostion \ref{prop:discrimmap} are products of those appearing in Proposition \ref{prop:discrimmapsurjective}. The map of Propostion \ref{prop:discrimmap} is thus surjective if the cohomology groups are non-trivial. This is the case precisely if at least one factor of $E_i$ over $E_i^\sigma$ is not split.
\end{proof}

\begin{example}
Fix a prime numbers $p_1,p_2$ and $p_3$ such that $\left(\tfrac{p_1}{p_3}\right) = 1$ and $\left(\tfrac{p_2}{p_3}\right) = -1$.
Set $A=\QQ(\sqrt{p_2})$ with the unique non-trivial involution, and $E=\QQ(\sqrt{p_2},\sqrt{p_1p_2})$ with the unique involution extending $\sigma$ acting trivially on $\sqrt{p_1p_2}$.
Then $A_{p_3} = \QQ_{p_3}(\sqrt{p_2})$ and $E_{p_3} \simeq A_{p_3} \times A_{p_3}$ with $\sigma$ interchanging factors.
It follows that the torus $T_{E,\sigma}$ does not embed into any unitary group which is not quasi-split at $p_3$.
\end{example}

\begin{theorem}
Let $K$ be $\RR$ or $\CC$ and fix $A$ a quadratic extension of $K$ (potentially split).
Given $H$ a Hermitian form and $(E,\sigma)$ an \'etale algebra with involution (of the second kind) over $A$. Then there exists $\lambda \in E^\sigma$ with:
\[ H \simeq H_{E,\sigma,\lambda} \]
if and only if 
\TODO - phrase signature conditions
\end{theorem}
\begin{proof}
If $A$ is split over $K$ then there is a unique Hermitian space and hence nothing to show.
This is the case unless $A=\CC$ and $K=\RR$. In this case $E$ is $\CC^n$, however some of the factors might be interchanged by $\sigma$.
By Proposition \ref{prop:signature} the factors interchanged by $\sigma$ must contribute $(1,1)$ to the signature.
Wheras those not interchanged contribute $(1,0)$ or $(0,1)$ depending on the choice of $\lambda$.
From this we conclude the result.
\end{proof}

\begin{example}
\TODO - A CM-field field $A$ and a complex field $E$ such that $E\otimes_\QQ \RR$ has factors interchanged by an involution.
\end{example}


\begin{lemma}\label{Lem:liftsigma}
Let Let $K$ be a number field and fix $A$ a quadratic extension of $K$ (potentially split).
Fix $(E,\sigma)$ an \'etale algebra with involution (of the second kind) over $A$.
Write $E = E_s \times E_{ns}$ where $E_s \simeq E_s^\sigma \times E_s^\sigma$ and $\sigma$ interchanges factors and $E_{ns}$ is a product of $\sigma$-stable fields.
There exists $\tilde{\sigma} \in \Gal(\overline{K}/K)$ inducing $\sigma$ on $E_{ns}$ (that is a unique $\tilde{\sigma}$ induces the involution on each factor.
\end{lemma}
\begin{proof}
Write:
\[ E_{ns} = \prod E_i \]
and set $\tilde{E} = E_1^\sigma E_2^\sigma\ldots E_{n}^\sigma$ to be the composite field of all the fixed fields of the $E_i$.
Notice that $E_i \simeq E_i^\sigma\otimes_K A \simeq E_i^\sigma A$.
Now, $A$ and $\tilde{E}$ are algebraicly disjoint fields and for all $i$ there exists canonical embeddings of fields:
\[ \xymatrix{ E_i \simeq E_i^\sigma\otimes_K A \ar@{^(->}[r]\ar@{-}[d] &  \tilde{E} \otimes_K A \simeq \tilde{E}A \ar@{-}[d] \\ 
                      E_i^\sigma \ar@{^(->}[r]\ar@{-} & \tilde{E}.}
 \]
The field $\tilde{E}A$ has a unique involution $\tilde{\sigma}$ such that $(\tilde{E}A)^{\tilde{\sigma}} = \tilde{E}$. Since $E_i^{\tilde{\sigma}} = E_i^\sigma$ this involution restricts to $\sigma$ on the image of $E_i$.
Finally, $\tilde{\sigma}$ lifts to an element of $\Gal(\overline{K}/K)$, giving us the claim.
\end{proof}

\begin{theorem}
Let $K$ be a number field and fix $A$ a quadratic extension of $K$ (potentially split).
Given $H$ a Hermitian form and $(E,\sigma)$ an \'etale algebra with involution (of the second kind) over $A$. Then there exists $\lambda \in E^\sigma$ with:
\[ H \simeq H_{E,\sigma,\lambda} \]
if and only if there exists $\lambda_\fp$ for all localizations $\fp$ of $K$ (both archimedian and non-archimedian).
\end{theorem}
\begin{proof}
\TODO - complete sketch (also check that this is true).
\TODO - need to distinguish primes of $K$, $A$ and $E$.

Write $E^\sigma = \prod_{i=1}^n E^\sigma_i$.
We have the following diagram:
\[
\xymatrix{
(E^\sigma)^\times/N_{E/E^\sigma}(E^\times) \ar[r]^{(\cdot,z)} \ar[d]& H^2(E^\sigma, \pm 1) \ar[r] \ar[d]& \underset{E_{i,\fp} \text{ not split}}\oplus H^2(E^\sigma_{i,\fp},\pm 1) \ar[d]\\
K^\times/N_{A/K}(A^\times) \ar[r]^{(\cdot,\delta_{A/K})} & H^2(K, \pm 1) \ar[r] & \underset{A_\fp \text{ not split}}\oplus H^2(K_\fp,\pm 1)
}
\]
where the compositions of the horizontal maps are injective.
We must show that an element in $K^\times/N_{A/K}(A^\times) $ is in the image of the left verticle map, provided its image is in the image of the right hand verticle map.
We must do this in spite of the fact that the horizontal maps are not surjective.

The failure of surjectivity is captured by the following, the bottom map surjects onto the kernel of the map:
\[ \underset{A_\fp \text{ not split}}\oplus H^2(K_\fp,\pm 1)  \rightarrow \{ \pm1\} \]
given by $(x_\fp) \mapsto \prod x_\fp$. The top map surjects onto the kernel of the map:
\[ \underset{E_{i,\fp} \text{ not split}}\oplus H^2(E_{i,\fp}^\sigma,\pm 1)  \rightarrow \{ \pm1\}^n \]
given by $(x_{i,\fp}) \mapsto (\prod x_{1,\fp}, \ldots, \prod x_{n,\fp}) $.

To complete the argument, it suffices to show that the kernel of the map:
\[\underset{E_{i,\fp} \text{ not split}}\oplus H^2(E_{i,\fp}^\sigma,\pm 1)  \rightarrow \underset{A_\fp \text{ not split}}\oplus H^2(K_\fp,\pm 1) \]
surjects onto the kernel of the map:
\[ \{ \pm1\}^n \rightarrow \{ \pm1\}. \]
To show this, it suffices to do this for a single fixed $\fp$ a prime of $K$.
Fix $\fp$ any prime of $K$ where $\tilde{\sigma}$ of Lemma \ref{Lem:liftsigma} acts as Frobenius. Then for all $i$ there exists $\fp_i | \fp$ prime of $E_i^\sigma$ for which $E_{\fp_i,i}$ is non-split. It follows that $\oplus_i H^2(E_{\fp_i,i}^\sigma,\pm1)$ surjects onto $\{\pm 1\}^n$, thus this also holds for the larger group, and hence also for the kernels.
\end{proof}

\begin{theorem}
\TODO - summarize a pure global existance statement.
\end{theorem}

\begin{remark}
\TODO - Can phrase this as: The reflex algebra splits the even clifford algebra + signature conditions.
Is there a meaningfull interpretation of the clifford algebra in this context?
Can we interpret it as telling us about a quaternioninc symplectic group embedding of the unitary group?
\end{remark}




\section{Properties of $T_{E,\sigma}$}

\subsection{Anisotropic}

\begin{proposition}
The torus $T_{E,\sigma}$ is anisotropic if all the field factors of $E$ are $\sigma$-stable.
\end{proposition}
\begin{proof}
\TODO-
That the condition is necissary is reasonably clear, sufficiency is less obvious.
\end{proof}

\begin{example}
Consider $E = A \times A$ where $\sigma$ acts on each factor.
This has the structure of an $A$ algebra, where $\sigma$ restricts to $\sigma$, by the inclusion $A\hookrightarrow A\times A$:
\[ a \mapsto (a,a) \]
The torus $T_{E,\sigma}$ is anisotropic.

\bigskip
Consider $E = A \times A$ where $\sigma$ acts to interchange factors.
Give this the structure of an $A$ algebra, where $\sigma$ restricts to $\sigma$, by the inclusion $A\hookrightarrow A\times A$:
\[ a \mapsto (a,\sigma(a)) \]
The torus $T_{E,\sigma}$ is not anisotropic.
\end{example}


\subsection{Unramified}

\begin{proposition}
The torus $T_{E,\sigma}$ is unramified if all the field factors of $E$ are unramified
\end{proposition}
\begin{proof}
\TODO-
\end{proof}


\subsection{``Relatively'' Unramified}

Your special unramifiedness when $A/K$ was ramified.

\section{The Building of the Unitary Group}

\subsection{The reduction of $T_{E,\sigma}$}

\section{Cohomological Reinterpretation}

This is stuff I don't recall super well, I think the following is roughly true, you might recall better than I.

\begin{question}
The set of pairs $(T,U)$ consisting of $T$ a maximal torus (up to rational conjugacy) of $U$ and $U$ a pure inner form of a unitary group are in bijection with:
\[ H^1( \Gal(\overline{K}/K), N_{U_0}(T_0) ) \]
\end{question}
If $T_0$ is split this becomes:
\[ H^1( \Gal(\overline{K}/K), W(U,T_0) ) \]

\begin{question}
The set of pure inner forms of $U$ containing a torus isomorphic to $T_0$ is in bijection with the kernel of the natural map:
\[ H^1( \Gal(\overline{K}/K), N_{U_0}(T_0) ) \rightarrow H^1( \Gal(\overline{K}/K),N_{GL_n}(T_E )) \]
\end{question}

\begin{question}
the set of maximal tori $T$ in $U_0$ (up to rational conjugacy) is in bijection with the kernel of the natural map:
\[  H^1( \Gal(\overline{K}/K), N_{U_0}(T_0) ) \rightarrow H^1( \Gal(\overline{K}/K), U_0) \]
\end{question}

The above should somehow give us that:
\begin{question}
The forms of $T$ contained in $U$ are in bijection with:
\[ H^1(\Gal(\overline{K}/K), W(U_0,T_0)^{\Gal(\overline{K}/K} ) \]
\end{question}
However this could plausibly require that $T_0$ be split or that $U_0$ is quasi-split.










\end{document}